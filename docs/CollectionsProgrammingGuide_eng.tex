
% $Id: CollectionsProgrammingGuide_eng.tex,v 1.10 2001-07-05 13:15:55 yulp Exp $

\subsection{ Introduction }

 Query Collection expose high-level object model, intendent for 
work with data sets, organized as so-called collections.

 Using of Query Collections allow to reduce low-level SQL programming for
 accessing and modifying of data: all typical operations are implemented
 in collections objects.

\subsection{ General description  }

\subsubsection{ ``Phisycal sense'' of collection }

 What is collection inside - just a set of SQL sentences for reading, writing
and modification of data, which are evaluated during call of collection object methods.

 Example: if you want to select some data from one table, you use
some SQL sentence which have form:

\begin{verbatim}
 SELECT <select-part> FROM <from-part> WHERE <where-part>
\end{verbatim}

So this SQL sentence defines some data set, which can be described by
 parts of our SQL sentence.
Imagine now, that we want to retrieve from this data set all records, which
satisfy some condition.To do it we can use such methods from typical collection interfaces 
as \verb|retrieve_by_filter|; in SQL we must evaluate SQL sentence,
which looks as 
\begin{verbatim}
 SELECT <select-part> FROM <from-part> WHERE <where-part> AND <condition>
\end{verbatim}
 Method \verb|update_by_filter| (intendent to update data, which 
satisfy some condition ) causes evaluating of next SQL sentence:
\begin{verbatim}
 UPDATE <select-part> SET <set-part> WHERE <where-part> AND <filter>
\end{verbatim}
Note, that \verb|<set-part>| can be automatically 
deduced from \verb|<select-part>|.

I. e. during call of collection methods appropriate SQL sentences are
built and evaluated.

 So, what is the query collection itself: object, which simply
 keeps and evaluates set of SQL sentences for data access and
 modifying.


In more formal terms we can say, that
 UAKGCollection is defined by:
 \begin{itemize}
   \item Set of data fields (i. e. SELECT part of SQL sentence)
   \item Data Source (i. e. FROM part of SQL sentence)
   \item Query condition (i. e. WHERE part of SQL sentence)
   \item Ordering (i. e. ORDER-BY part of SQL sentence)
 \end{itemize}


\subsection{ Steps for using collections interfaces  }

 For work with query collections, application programmer must
perform the next steps:

 \begin{itemize}
   \item create Collection using appropriate methods of QueryManager.
  \item Using UAKGCollection you can query or modify data in it, or
    receive \verb|Iterator| for sequential access to data set.
  \item Before end of work it is necessary to clear server resources,
   associated with  UAKGCollection by call of \verb|UAKGCollection::destroy| 
 \end{itemize}

\subsection{ Creation of Query Collections }

 UAKGQuery supports two types of collections: 
 UAKGCollection and UAKGKeyCollection.

 UAKGCollection represents data set, UAKGKeyCollection - dataset with unique
keys.
 \newline
 \newline
 Exists 2 ways of creating UAKGCollection:
\begin{itemize}
   \item by SQL sentence
   \item by set of field names and conditions. (i. e. by parts)
 \end{itemize}

for example, let's look at next two code fragments:

\begin{verbatim}
 UAKGCollection_var collection = queryManager->create_collection_by_parts(
                                          "F1,F2", "UAKGTEST", "F1=1", "F2");
\end{verbatim}

or

\begin{verbatim}
 UAKGCollection_var collection = queryManager->create_collection_by_query(
                         "select F1,F2 from UAKGTEST where F1=1 order by F2");
\end{verbatim}

 This 2 fragments give us identical result: collection which consists 
 of fields F1 and F2 of records in UAKTEST, ordered by F2, where F1 is 1.
 
 Parameters of \verb|create_collection_by_fields| in first code fragment are:

 \begin{enumerate}
   \item set of fields: SELECT part of SQL sentence
   \item data source: FROM part.
   \item select condition: WHERE part.
   \item ordering (optional) : ORDER part.
 \end{enumerate}

\subsection{ Data Access }

\subsubsection{ Receiving data with the help of Iterator's interface }

 You can use \verb|Iterator| concepts to retrive data.
 The steps for work with iterator are described below:

 \begin{enumerate}
  \item receive \verb|Iterator|, by call of \verb|create_iterator| method of collection.
  \item Use \verb|Iterator| API for navigation across data set.
  \item destroy \verb|Iterator|
 \end{enumerate}

Example:

\begin{verbatim}
 UAKGIterator_var iterator_ = collection_->create_iterator();
 Boolean more = true;
 while( more )
 { 
   OctSeq_var octSeq_ = iterator_->fetch_rc(50, more);
   printRC( octSeq_ );
 }
 iterator->destroy();
\end{verbatim}

 Interface \verb|Iterator| expose next methods:

 \begin{itemize}
   \item \verb|fetch_rc( ULong n, Boolean& more )|
     Read \verb|n| records as byte stream in RC-coding
   \item \verb|fetch_records( ULong n, Boolean& more )|
     Read \verb|n| records and return it as RecordSequence.
   \item \verb|skip( ULong n, Boolean& more )|
     Skip \verb|n| records without retrieving of data.
 \end{itemize}

\subsubsection{  Retryiving data via collection methods }

 You can read data from UAKGCollection
 directly, using next collection methods:

\begin{itemize}
 \item \verb|retrieve_by_filter( const char* where_filter )|
 \item \verb|retrieve_by_pattern( const Record& pattern )|
\end{itemize}

Both of those methods return RC-coded byte stream, which contains records
from data set collection filtered by parameter.
First method filters data set before retrieving by logical expression in \verb|where_filter|, second -- by principle of pattern matching (see \ref{patterns} ).

Example:

\begin{verbatim}
 UAKGCollection_var collection_ = queryManager->create_collection_by_query(
                          "select F1,F2 from UAKGTEST where F1=1 order by F2");
 OctSeq_var octSeq_ = collection_->retrieve_by_filter("F2='test'");
 printRC( octSeq_ );
 collection_->destroy();
\end{verbatim}

 This code fragment will print all records of created collection, where
F2 is equal to string \verb|'test'| . (I. e. all records in result will
have form \verb|(1,'test')| ).


\subsubsection{ Pattern matching }
 \label{patterns}

 Concept of pattern matching is often used in UAKGQuery Service for filtering
 data sets: for example, method \verb|retrieve_by_pattern| retrieves records
 which  match pattern, passed as argument of this method. Which means:
 pattern matching concept is derived from QBE (Query By Example) concept --
 in result set we receive records, which are "the same" as pattern, exclude
 fields, which set in pattern as empty.

More formal:

 Pattern is a record (i. e. have type \verb|Record|). Fields of pattern are
used for filtering data sets. 
Pattern matching for collection is defined as following:
\begin{itemize}
 \item Structure of pattern record  must be equal to structure of records in queuing datatset (i.e. number of fields and types must be equal)
 \item Record \verb|r| match pattern \verb|p| $iff$
 \begin{enumerate}
   \item $\forall i \in 0\dots length(r): p[i]!=NIL \rightarrow p[i]\; match \;r[i]$
    (i. e. for all field indexes, appropriate fields must match if pattern 
 field is not NULL).
   \item For fields $p_i,r_i$, $p_i\; match \;r_i$ iff
    \begin{enumerate}
       \item $type(r_i)==string \rightarrow (r_i\;\;LIKE\;\;p_i)$ 
       \item $type(r_i)!=string \rightarrow (r_i\;\;==\;\;p_i)$ 
    \end{enumerate}
    I. e. 2 fields match, if they are identical for non-string types, or
    if  LIKE matching is met for string types.
 \end{enumerate}
\end{itemize}

Practical example:\\

Suppose we have table UAKGTEST with two fields: F1 of type NUMBER and F2 of type VARCHAR2.
Then, we created collection which works with records of this table.
Then, for retrieving all records from this collection with F1=5 programmer must perform next steps:
\begin{itemize}
 \item Form pattern as a 2 field record , where first field must be 5, second - NULL . (i. e. second field  will not participate in matching).
 \item get needed records as result for \verb|retrieve_by_pattern| with pattern as parameter.
\end{itemize}

\begin{verbatim}
Record_var pattern = new Record;
pattern->length(2);
FieldValueAccess::setLong(pattern[0],5);
FieldValueAccess::setNull(pattern[1]);
OctSeq_var octSeq = collection_->retrieve_by_pattern(pattern);
printRC(octSeq);
\end{verbatim}

\subsection{ adding, updating and deleting of data }

  Of course, except retrieving of data, you can add, modify or delete collection items.

 List of aprropriate methods:
 \begin{enumerate}
  \item Adding data:
     \begin{itemize}
       \item \verb|add_record( const Record& element )| -- add one record to collection
       \item \verb|add_records( const RecordSeq& elements )| -- add sequence of records to collection.
       \item \verb|add_rc( const OctSeq& rc )| -- add sequence of records, coded in RC stream. 
     \end{itemize}

Note, that during adding of records no checking of belonging added data to 
target collection is performed.
For example, let's look at next code fragment:
\begin{verbatim}
 UAKGCollection_var collection=qm->create_collection(
                                         "select * from emp where deptno=22"
                                                    );
 ULong n=collection->number_of_records();
 cout << "now number of records is:" << n << endl;
 RecordDescription_var rd = collection->get_record_description();
 Record_var record=CosQueryFacade::RecordAccess::createRecordByDescription(rd);
 CosQueryFacade::RecordAccess::setShortByName(record.inout(),"EMPNO",11,rd);
 CosQueryFacade::RecordAccess::setShortByName(record.inout(),"DEPTNO",11,rd);
 collection->add_record(record);
 n=collection->number_of_records();
 cout << "now number of records is:" << n << endl;
\end{verbatim} 
 After successfull execution of this fragment one item will be added to \verb|emp| table, but number of elements in collection will not change.

  \item Updating of data:
     \begin{itemize}
       \item \verb|update_by_pattern( const Record& newRecord, const Record& pattern )| -- set to \verb|newRecord| records, which matching pattern \verb|pattern|. For example, next code fragment:
\begin{verbatim}
  SRecord sr1, sr2;
  collection_->update_by_pattern(
                 sr1._short(1)._short(2).in(), sr2_._nil()._short(2)
                               )
\end{verbatim}
    
       \item \verb|update_by_filter( const Record& newRecord, const char* filter )| - set to \verb|newRecord| all records, which satisfy logical condition in \verb|filter|.

For example, next code fragment:
\begin{verbatim}
   collection->update_by_filter(sr._short(1)._short(2),"x2=2");
\end{verbatim}
    will set to \verb|(1,2)| all records in collection for which \verb|x2=2|
     \end{itemize}
  \item Deleting data:
     \begin{itemize}
       \item \verb|remove_record( const Record& record )| - remove records, which are equal to given record.
       \item \verb|remove_records_by_filter( const char* filter )| - remove records by filter.
       \item \verb|remove_records_by_pattern( const Record& pattern )| - remove records by pattern.
       \item \verb|remove_all_records()| - remove all records from collection.
     \end{itemize}
 \end{enumerate}

Example:

\begin{verbatim}
  UAKGCollection_var collection_ = queryManager->create_collection_by_query(
                           "select F1,F2 from UAKGTEST where F1=1 order by F2");
  Record_var inpRecord = new Record;
  inpRecord->length(2);
  FieldValueAccess::setLong( inpRecord[0], 5);
  FieldValueAccess::setString( inpRecord[1], "test" );
  collection_->add_record(inpRecord);
  FieldValueAccess::setNull( inpRecord[1] );
  OctSeq_var octSeq_ = collection_->retrieve_by_pattern(inpRecord);
  printRC( octSeq_ );
  collection_->remove_records_by_pattern( inpRecord );
  collection_->destroy();
\end{verbatim}

\subsection{ Collection queries }

 The shown up set of methods is not always  sufficient for comfortable access to data collection.
Sometimes we need to perform more complicated actions:
for example change set of requested fields or receive some data from
linked table.

 For this purpose family of \verb|evaluate_xxx| methods of \verb|UAKGColletion| is intendent. (i. e. collections implements interface QueryEvaluator).

 For example, next query:
\begin{verbatim}
 result = collection->evaluate_rc_e("select x1,x3 from @ where x1=x2");
\end{verbatim}
 will fetch in result fields \verb|x1| and \verb|x2| for records, in which \verb|x1=x3|.

Next code fragment:
\begin{verbatim}
 collection=queryManager->create_collection("select * from orders");
 result=collection->evaluate_rc( 
   "select @,CUSTOMERS.NAME from @,CUSTOMERS where CUSTOMER.ID=customer_id
                              );
\end{verbatim}
 will retrieve all fields of order table and names of customers from linked table.  

Next query:
\begin{verbatim}
 result=collection->evaluate_rc("select @ from @ where not @")
\end{verbatim}
 will invert collection (i. e. select all records which are situated at the
same data source, but not belong to original collection).

Now, lets define semantics of our extending of SQL by \verb|@| :
\newline
If collection is based on SQL expression in form:
\begin{verbatim}
  select <select-part> from <from-part> where <where-part>
\end{verbatim}
and we pass in \verb|evaluate_xxx|  expression:
\begin{verbatim}
  select @,<new-select-part>
   from @,<new-where-part> [where <new-where-part>]
\end{verbatim}
 than result expression will have form:
\begin{verbatim}
  select <select-part>,<new-select-part>
   from <from-part>,<new-from-part>
   where (<where-part>) AND (<new-where-part>)
\end{verbatim}

 Suppressed SQL sentences is handled in less or more equal way.
 
\subsubsection{ Limitations }

  \begin{itemize}
    \item Colection can evaluate only SELECT queries without \verb|HAVING_BY| and \verb|GROUP_BY| clauses. Full grammar of Collection SQL is listed in 
  \ref{Collection-select-grammar}
  \end{itemize}

\subsubsection{ List of methods }

\begin{itemize}
 \item 
   \begin{verbatim}
      evaluate_rc( const char* queryText, const char* queryFlags, 
                   const RecordDescription& recordDescription, 
                   const OctSeq& params)
   \end{verbatim}
 \item 
   \begin{verbatim}
     evaluate_rc_inout(const char* queryText, const char* queryFlags, 
           const RecordDescription& recordDescription, OctSeq& params)
   \end{verbatim}
 \item 
   \begin{verbatim}
     evaluate_record( const char* queryText, const char* queryFlags, 
           const RecordDescription& recordDescription, const Record& params )
   \end{verbatim}
 \item 
   \begin{verbatim}
    evaluate_records_inout(const char* queryText, const char* queryFlags, 
                           const RecordDescription& recordDescription, 
                           RecordSeq& params)
   \end{verbatim}
 \item 
   \begin{verbatim}
     evaluate_records(const char* queryText, const char* queryFlags, 
                      const RecordDescription& recordDescription, 
                      const RecordSeq& params)
   \end{verbatim}
\end{itemize}

\subsection{ SubCollection|}

 As can be deduced from name of sections:  you can query some subset of collection in new collection (so-called subcollection).

Application programmer must perform next steps to use subcollection technique:

\begin{enumerate}
  \item Creation of \verb|SubCollection|
     \begin{itemize}
       \item \verb|create_subcollection( const char* subquery )| 
       In this method \verb|subquery| is expression on the same language, as
       for \verb|evaluate_xxx| family of collection methods. Created collection
       is result of evaluating such query as collection. Note, that if 
        data-source (i.e. where-part of subquery) contains multiple tables,
        then resulting collection is read-only.
       \item \verb|create_subcollection_by_pattern( const Record& pattern )|
       The resulting collection is just subset of records of original collection, which match pattern \verb|pattern|.
     \end{itemize}
  \item Work with received subcollection, using UAKGCollection API
  \item delete \verb|SubCollection| using method destroy()
 \end{enumerate}

Example:

\begin{verbatim}
 UAKGCollection_var collection_ = queryManager->
              create_collection_by_fields("tname, tabtype", "tab", "1=1", "");
 UAKGIterator_var iterator = collection_->create_uakgiterator();
 Boolean more;
 RecordSeq_var recordSeq = iterator->fetch_records(0, more);
 FieldValueAccess::setNull(recordSeq[0][0]);
 iterator->destroy();
 UAKGCollection_var new_collection_ = collection_->
                                create_subcollection_by_pattern(recordSeq[0]);
 iterator = new_collection_->create_uakgiterator();
 recordSeq = iterator->fetch_records(0, more);
 printRecordSeq(cout,recordSeq.in());
 iterator->destroy();
 new_collection_->destroy();
 collection_->destroy();
\end{verbatim}


\subsection{ UAKGKeyCollection}

 UAKGKeyCollection is a specialization of UAKGCollection with property
``have primary key''. \footnote{i. e. key, which is unique for each record in collection }.

 Additional methods provide API for accessing, modifying and deleting elements
by keys.

\subsubsection{ Creation of UAKGKeyCollection }

 for creation of KeyCollection you can use 2 methods of UAKGQueryManager:
\begin{itemize}
  \item \verb|create_key_collection_by_parts| -- which is create key collection,by appropriative parts of SQL sentence and yet ine additional part: list of field names, of which key is consists.
Example:
\begin{verbatim}
 UAKGKeyCollection_var collection = queryManager->
   create_key_collection_by_parts("F1,F2","UAKGTEST","F1=1","F2","F1");
\end{verbatim}
 \item \verb|create_key_collection_by_query| -- which is create key collection by SQL query , with yet one our extension of SQL: WITH KEY clause.
\begin{verbatim}
 UAKGKeyCollection_var collection = queryManager->
   create_key_collection_by_query(
    "select F1,F2  from UAKGTEST where F1=1 order by F2  with key F1"
                                 );
\end{verbatim}
  \verb|WITH KEY| clause can be used only for creation of key collection and
SQL sentence with it must have following syntax:
\begin{verbatim}
  SELECT <selection> <table-expr> WITH KEY <selection>
\end{verbatim} 
\end{itemize}

 As in SQL92 key can be compound:
\begin{verbatim}
  UAKGKeyCollection_var collection=queryManager->
           create_key_collection_by_query(
                     "select * from X with key x1, x2");
\end{verbatim}

-these two parts of code are equivalent,  they create the collection UAKGTest, where F1 is the primary key.  As we can see, we have one more SQL- clause\verb|WITH KEY|.
It may be used only to get collection with keys, enclosure syntax with this clause is following:

\begin{verbatim}
 SELECT <selection> <table-expr> WITH KEY <selection>
\end{verbatim}

As in SQL92,  the sequence of margins can create the key too:
 \begin{verbatim}
 UAKGKeyCollection_var collection = queryManager->
      create_key_collection_by_query(
                 "select * from X with key x1,x2");
\end{verbatim}


 Note, that specification of correct keys is business of application 
programmer : UAKGQueryService use this information, but does not do
any checking for accordance with real structure of database.

\subsubsection{ Methods of UAKGKeyCollection }

 UAKGKeyCollection provide next methods, for retrieving, updating
 and deleting  elements by keys:

 \begin{itemize}
   \item \verb|retrieve_record_with_key(const Record& key)| - retrieve record with key \verb|key|.
   \item \verb|retrieve_records_with_keys(const OctSeq& keys)| - retrieve sequence of records with accordance of \verb|keys|
   \item \verb|update_record_with_key(const Record& newRecord, const Record& key)| -- set to \verb|newRecord| element with key \verb|key|.
   \item \verb|update_records_with_keys(const OctSeq& records)| -- update records, with the same keys, as appropriate records in argument.
   \item \verb|remove_records_with_keys(const OctSeq& keys)| -- remove records with keys.
 \end{itemize}

 Also 2 helper methods are provided by UAKGKeyCollection:

 \begin{itemize}
   \item \verb|get_key_description()| -- return description of collection key
   \item \verb|extract_keys(const OctSeq& records)| -- extract keys from sequence of records.
 \end{itemize}

Example:

\begin{verbatim}
  UAKGKeyCollection_var collection_ = queryManager->create_key_collection_by_query("select F1,F2 from UAKGTEST where F1=1 with key F2");
  Record_var inpRecord_ = new Record;
  inpRecord_->length(1);
  FieldValueAccess::setString(inpRecord_[0], "test");
  collection_->remove_record_with_key( inpRecord );
  collection_->destroy();
\end{verbatim}

\subsection{ Using of UAKGCollectionListener }

 It becomes common and useful technique to organize program coupling via 
 notifivation about important events in life of some service.
 On this purpose UAKGCollection provide
 such mechanism of "Listeners" - user can add to collection own implementation
 of \verb|UAKGCollectionListener| callback interface, which would be called
 during performing of collection actions. 

\begin{verbatim}
 interface UAKGCollectionListener
 {
   // called, when elements are added:
   void elements_added(in OctSeq elements);

   //called, when elements updated
   void elements_updated(in OctSeq prev_elements,
                         in OctSeq new_elements); 

   //called, when elements removed:
   void elements_removed(in OctSeq elements)

   //called, when all ellements in collection are removed
   void all_elements_removed();

   //called, when collection is destroyed
   void collection_destroyed();
 }
\end{verbatim}

 Application programmer can implement this interface and bind it with
 collection for receiving events via method:
\begin{verbatim}
 unsigned long UAKGCollection::add_listener(
                                 in UAKGCollectionListener listener,
                                 in unsigned short eventMask);
\end{verbatim}
 Method returns number, which identifiy passed listener from
collection side. This number can be used for unbinding listener from
collection with help of method:
\begin{verbatim}
 unsigned long UAKGCollection::remove_listener(in unsigned long listenerIndex);
\end{verbatim}

 At last, \verb|eventMask| is a bit mask of events, which listener want to receive.


 Using notification technique remember, that cost of collection data for 
notificating can be high, and that synchronized calls of callback functions
on each event are not scalable technique. If you have situation, when N service
clients must receive notifications, than better do not register N listeners but 
create additional element of infrastructure, which receive notification and 
send it to clients, using asynchronics techniques (for example, via CORBA Event Service).


\subsection{ Limitations of using UAKGQuery collections interfaces }

For present tim , one instance of collection can be used in only 
one instance of transaction at the same time.

(I. e. concurrent access to collection from different transactions is 
not safe).


