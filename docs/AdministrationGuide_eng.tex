
\documentclass[10pt]{article}
\usepackage[OT2,T2A]{fontenc}
\usepackage{graphicx}
\usepackage{verbatim}

\title{ UAKGQuery: Administration Guide }

% $Id: AdministrationGuide_eng.tex,v 1.19 2002-01-29 18:30:19 kav Exp $

\begin{document}

\maketitle{}
DocumentID: GradSoft-AD-19/09/2000-1.2E

\tableofcontents

\section{ Introduction }

UAKGQuery is a  CORBA service which proviades CORBA API for  access to relational database.
 This release supports Oracle as relational database.

 Using UAKGQuery is possible to organize effective and uniform data access
from different sources in heterogene computing environment.

\section{ Compilation }

\subsection{ Software needed: }
 
\subsubsection{ for Windows NT }

\begin{itemize}
 \item ORB - ORBacus-4.0 or higher
 \item C++ Compliler: Microsoft Visual C 6.0 or higher. 
 \item Database server: Oracle 8.1.5  or higher (needed only for compilation of server part)
 \item Database server: InterBase 6 or higher (needed only for compilation of server part)
 \item GradC++ ToolBox 1.4.0 or higher.
\end{itemize}


\subsubsection{ for Unix }

\begin{itemize}
 \item ORB - one of
  \begin{itemize}
     \item ORBacus-4.0 or higher, 
     \item TAO-1.1, TAO-1.2 or higher
   \end{itemize}
 \item C++ compiler : gcc-2.95.2 or SunPro CC 4.2 
 \item Database server: Oracle 8.1.5 or higher (needed only for compilation of server part)
 \item Database server: InterBase 6 or higher (needed only for compilation of server part)
 \item GradC++ ToolBox 1.4.0 or higher.
\end{itemize}

\subsubsection { Software needed for transaction support }

 If you want to use transactional features of \verb|UAKGQuery|, ORBacus
 Transaction service and OOC \footnote{now iona} ORBacus ORB is needed.

 Note: transactional features are absent subject to \verb|UAKGQuery|
 has been compiled for Windows NT.

\subsection{ Compilation and Installation }

\subsubsection{ For Windows NT }

\begin{enumerate}
 \item Unpack archive UAKGQuery-1.1.0.zip
 \item Change current directory to CosQuery.1
 \item Edit file \verb|env_inc.nt.mak| and set next varibles in it: 
      \begin{itemize}
        \item  PROJECT\_ROOT - <full\_path\_to\_CosQuery.1>\verb|\|CosQuery.1
        \item  ORB\_DIR - directory, where ORBACUS is installed.
        \item  OCI\_HOME - directory, where Oracle is installed (uncomment line HAVE\_ORACLE=1).
        \item  IBASE\_HOME - directory, where Interbase is installed (uncomment line HAVE\_INTERBASE=1).
        \item  MSVC\_HOME - directory, where MS Visual C++ is installed.
        \item  INSTALL\_DIR - absolute path to directory, where you want to install UAKGQuery. (warning: C++ ToolBox has to be installed there)
      \end{itemize}
 \item Change directory to CosQuery.1 
 \item Start compilation:
   \begin{itemize}
     \item for server: start make.bat 
     \item for clent: start make.bat client
   \end{itemize}
 \item Start installation process:
   \begin{itemize}
     \item for server: run make.bat install
     \item for client: run make.bat install-client
   \end{itemize}
\end{enumerate}

\subsubsection{ For Unix }

\begin{enumerate}
 \item Unpack archive UAKGQuery-1.1.0.tar.gz
 \item Change directory to CosQuery.1
 \item Start ./configure. Set configure options if needed.
 \begin{itemize}
   \item \verb|--with-ots| - if you want to use CORBA Transaction Service.
   \item \verb|--with-ob-xa| - if you want to use XA extensions of CORBA Transaction Service .
   \item \verb|--toolbox=<path-to-toolbox>| - if you installation directory of 
       GradSoft C++ toolbox  differs from installation directory of UAKGQuery
       (default is /usr/local)
   \item Also you can set location of ORB and other packages: use command
   \verb|./configure --help| to see list of all possible options.
 \end{itemize}
  
 \item Start compilation:
   \begin{itemize}
     \item for server: run gmake 
     \item for client: run gmake client-stub-library
   \end{itemize}
 \item Start installation:
    \begin{itemize}
      \item for server: run make install
      \item for client: run make install-client-stub-linrary
   \end{itemize}
\end{enumerate}


 After installation next files will be installed into \verb|INSTALL_DIR|
 directory (which is defined as variable \verb|INSTALL_DIR| in NT Makefile 
 or as configuration parameter \verb|prefix| for UNIX)

\begin{enumerate}
  \item
      \begin{enumerate}
      \item idl-modules: \\
            CosQuery.idl \\
            CosQueryCollection.idl \\
            CosQueryIDLConfig.idl \\
            CosQueryIDLConfigV2.idl \\
            RC.idl \\
            UAKGQuery.idl\\
            - to subdir \verb|<install-dir>/idl|  
      \item
            header files of CORBA stubs and skeletons, received 
            during compilation of IDL files by IDL compiler ,
            \\
            - to {\bf include} subdir
      \item
            header files \\
            DecimalAccess.h \\          
            FieldDescriptionAccess.h \\
            FieldValueAccess.h \\
            RecordAccess.h \\
            RecordDescriptionAccess.h \\
            - to {\bf include/CosQueryFacade} subdir
      \item
            library UAKGQueryClient.lib (for Windows NT) or libUAKGQueryClient.a (for UNIX)\\
            - to {\bf lib} subdir
      \item
            binary UAKGQueryService \\
            - to {\bf bin} subdir.
      \end{enumerate}
\end{enumerate}


\section{ Using }

\subsection{ Starting of server }

 You must set network port for requests listening in Server configuration
parameters.

 The way of doing this is ORB-depended, it can be done from server command
 line or from ORB configuration file: consult with your ORB manual for details.
 
 Below is list of such options for common object request brokers:

\begin{itemize}
 \item For ORBacus use option \verb|-OAport <port>| or property ooc.iiop.port in
configuration file.
 \item For OmniORB use option \verb|-ORBpoa_iiop_port <port>|  
 \item For TAO-1.1 use option \verb|-ORBport <port>|
 \item For TAO-1.2 use option \verb|-ORBEndPoint iiop://:<port>|
\end{itemize}

 Another precondition for succesfull service start is existance of installed
Oracle8 client in working environment.

\subsubsection{ Logging  }

 UAKGQuery use standart UNIX syslog interface for logging. So, consult with
you operation system documentation for details, next example will work
with most of supported UNIX systems:
\begin{itemize}
\item Add in file /verb|/etc/syslog.conf| next strings:
\begin{verbatim}
!UAKGQueryService
*.*			/var/log/UAKGQuery.log
# in addition call administrator on fatals. 
*.fatal			|mail -s "UAKGQuery fatal" admin@his.cell.phone.kiev.ua
\end{verbatim}
\item Create empty file \verb|/var/log/UAKGQuery.log|
\end{itemize}

 In addition you can dup log output to file of to stderr by setting 
options of UAKGQueryService.

\subsubsection{ Command String options  }

\begin{itemize}  
 \item Command String ORB parameters, as written in you ORB manual.
 \item \verb|--ior-file-UAKGQueryService <fname>| - write  ior of UAKGQueryService to file with name {\bf fname}. 
 \item \verb|--ior-stdout| - output QueryService IOR to standart program output
 \item \verb|--with-naming| - register UAKGQueryService in CORBA Naming Service (note, that CosNaming initialization parameters must be aviable to ORB) .
 \item \verb|--config <fname>| - read configuration from file with name \verb|fname|.
 \item \verb|--ORACLE_XA <xa_string>| - use Oracle XA transaction Oracle.
       This option is aviable only for UNIX in the case of server was compiled with XA support. 
 \item \verb|--log-to-file <fname>| - dup log output to file \verb|<fname>|
 \item \verb|--log-to-strerr| - dup log output to UAKGQuery stderr.
 \item \verb|--log-sql| - log processed SQL sentences.
 \item \verb|--no-syslog| - does not use syslog for logging.
\end{itemize}

\subsection{ Starting of UAKGQueryService Client }

 When you start ordinary UQAKGQuery client, which receive reference on
UAKGQuery via resolving of initial references, you can use next command line
parameters for client ORB:
\begin{verbatim}
-ORBInitRef UAKGQueryService=\
            corbaloc::srv-host-name:port/UAKGQueryService
\end{verbatim}

where:
\begin{itemize}
  \item srv-host-name -- IP address of machine, where UAKGQueryService runs.
  \item port -- port, on which UAKGQueryService listenes for requests.
\end{itemize}


\section{ Changes }

\begin{enumerate}
  \item 24.01.2002 - \begin{enumerate}
                     \item installing procedure description for Windows NT
                           made to be adequate 
                     \item pointed, transaction service for Windows NT is absent
                     \end{enumerate}
  \item 21.01.2002 - version dependency check for new release
  \item 05.07.2001 - reflected 1.0.2 software requirements, added section about logging.
  \item 06.05.2001 - review.
  \item 26.04.2001 - review, corrections of NT install procedure.
  \item 12.07.2000 - first english edition.
\end{enumerate}

\end{document}
