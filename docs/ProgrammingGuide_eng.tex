
\documentclass[10pt]{article}
\usepackage[OT2,T2A]{fontenc}
\usepackage[koi8-u]{inputenc}
\usepackage{graphicx}
\usepackage{verbatim}

\title{ UAKGQuery: Programming Guide }

% $Id: ProgrammingGuide_eng.tex,v 1.26 2002-12-05 12:01:22 kav Exp $


\bibliographystyle{plain}

\begin{document}

\maketitle{}
DocumentID: GradSoft-PR-e-16/06/2000-1.5

\tableofcontents

\section{ Introduction }

 UAKGQuery is a CORBA \cite{OMG-CORBA-1} service intended for database
 access. It provides CORBA API for effective querying of relational
 databases and define high-level object interfaces for data access.

 With UAKGQuery it is possible to organize effective and uniform access
 to relational databases, based on high-level abstract model and independent
  from database service locations and database architecture limits.

 This document is unformal description of using UAKGQuery API. 
 For full formal reference, please, look at API guide.

 \subsection{History}

 UAKGQuery Service is in active development from 1998. 
 It grew from the implementation of CORBA Query Service
 \cite{OMG-Cos-Query}. After analyzing of scalability and performance
of original CosQuery API, structure of UAKGQuery Service was fully refined.
Some of methods applied for building of effective distributed applications
can be found in \cite{rssh-1} \cite{rssh-2}

 \subsection{ Knowledge Needed For Reading This Manual }

 We assume, that reader has basic knowledge of CORBA architecture
 \cite{OMG-CORBA-1} and has some experience with CORBA C++ application.
 (we recommend \cite{Adv-CORBA-Programming} as learning material).
 Also assumed, that reader knows what SQL is and what relational databases are.

 The last chapter requires knowing of such entities, as transactions, 
 CORBA Transaction Service \cite{OMG-COS-OTS} and XA transactions 
 \cite{XOPEN-XA}.  But you can omit this chapter during the first reading.


\section{ General description of UAKGQuery using  }

 
 Application programmer must perform next steps for  using UAKGQuery:
 
 \begin{itemize}
   \item Initialize ORB, by calling function \verb|CORBA::ORB_init|
   \item Initialize UAKGQueryService, by calling  function
  \verb|initUAKGQueryService|. This function is defined in header file
  \verb|UAKGQuery2Init.h|
   \item Receive object reference to DBConnectionManager
 (manager of database connection) via CORBA mechanism of resolving initial
 references.
   \item Create QueryManager, by calling \\ \verb|DBConnnectionManager::createQueryManager|, which expose API for creation and evaluation of
  SQL queries and so-called collections.
   \item With QueryManager it is possible whether directly evaluater SQL queries,
  or create Query or Collection objects for interactive reading/writing
  data.
   \item Before end of work it is necessary to disconnect from UAKGQuery
   service, by calling method \verb|destroy| of QueryManager.
 \end{itemize}

\section{ Connection To Database  }

 For connecting to database it is necessary to:
 \begin{enumerate}
   \item Receive initial object with name "UAKGQueryService", 
   \item Narrow it to type UAKGQuery::DBConnectionManager.
   \item receive QueryManager, calling DBConnectionManager::create
 \end{enumerate}

 This can be illustrated by next code fragment:
\begin{verbatim}
#include CORBA_H  // for CORBA
#include <UAKGQuery.h>  // for UAKGQuery
#include <UAKGQuery2Init.h>  // for UAKGQuery initialization

 ........

    // initialize ORB
    orb = CORBA::ORB_init(argc, argv);

    // initialize UAKGQueryService
    initUAKGQueryService(orb.in());

    // receive initial object
    CORBA::Object_var obj;
    try {
       obj = orb->resolve_initial_references("UAKGQueryService");
    }catch(const ORB::InvalidName&){
       cerr << argv[0] << ": can't resolve UAKGQueryService" << endl;
       return 1;
    }
        
    if(CORBA::is_nil(obj)) {
      cerr << argv[0] << ": UAKGQueryService is a nil object reference" << endl;
      return 1;
    }


    // narrow it to DBConnectionManager.
    DBConnectionManager_var dbManager = DBConnectionManager::_narrow(obj);
    if (CORBA::is_nil(dbManager)) {
      cerr << argv[0] << ": can't narrow dbManager to correct type" << endl;
      return 1;
    } 

   // receive QueryManager
    QueryManager_var queryManager;
    try {
      queryManager = 
           dbManager->createQueryManager("skott","tiger","Oracle","Oracle8","");
    }catch(QueryManagerNotFound){
      cerr << argv[0] <<": can't find query manager." << endl;
      return 1;
    }

    // now you can do something with QueryManager
    ..........
    ..........

    //time to disconnect.
    queryManager->destroy();

    // destroy ORB
    orb->destroy();

\end{verbatim}

\section{ Example of query evaluation }

 You can evaluate 
  queries use family of methods QueryEvaluator::evaluate 

    \verb|evaluate_rc|, \verb|evaluate_records|, 
    \verb|evaluate_record|, 
    \verb|evaluate_records_inout|, 
    \verb|evaluate_rc_inout|, 


 Those methods differ only by type of input and output parameters.
 Interface QueryManager is inherited from interface QuieryEvaluator, so you
 can apply this methods to QueryMeneger, which received from DBConnectionManager.

 Now we will show a simple example of such request, than it will start 
 more detailed description of methods and data types.

\begin{verbatim}

 try {

   OctSeq_var octSeq = queryManager->evaluate_rc("select * from tab","SQL92",
                                    RecordDescriptionAccess::empty(),
                                    RecordAccess::emptyOctSeq()); 
 }catch(QueryInvalid ex){
   cerr << "QueryTypeInvalid" << endl;
 }catch(QueryTypeInvalid ex){
   cerr << "QueryTypeInvalid" << endl;
 }catch(QueryProcessingError ex){
   cerr << "QueryProcessingError" << endl;
   cerr << ex.why << endl;
 }catch(const CORBA::SystemException& ex) {
#ifdef CORBA_SYSTEM_EXCEPTION_IS_STREAMBLE
   cerr << ex;
#else
   cerr << "System Exception" << endl;
#endif
 }

\end{verbatim}

 What this means:
 \begin{itemize}
  \item OctSeq -- This is a type of result, which means packed byte sequence.
  You can access to this sequence by using classes RCReader and RCWriter,
  described below.
  \item "select * from tab" -- text of SQL query.
  \item "SQL92" - query flags, where you can set query parameters such
  as language, number of prefetched rows and so on.
  \item RecordDescriptionAccess::empty() - constant of type RecordDescription:
   we pass descriptions of input query parameter there. As our query
   have no input parameters, than we pass empty RecordDescription.
   It is predefined in helper class RecordDescriptionAccess for 
   programmers pleasure.

\begin{footnote}
 so you don't write before any 
   query without parameters next piece of code:
\begin{verbatim}
  RecordDescription emptyDescription;
  emptyDescription.length(0);
\end{verbatim}
\end{footnote}

  \item RecordAccess::emptyOctSeq - input parameters of query. They must
  be in the same form, as results. As you see, RecordDescription::emptyOctSeq
  denotes empty octet sequence.

 \end{itemize}

 The next is function, which prints requested data:

\begin{verbatim}
void printRC(ostream& out, const OctSeq& octSeq)
{
 RCHeader header;
 CORBA::ULong pos=0;
 RCReader::readHeader(header,pos,octSeq);
 out << "Header: nRecords=" << header.nRecords << endl;
 out << "Header: nFields=" << header.nFields << endl;
 for(CORBA::Long nRecord=0; nRecord<header.nRecords; ++nRecord){
   for(CORBA::ULong nField=0; nField<header.nFields; ++nField){
     FieldValue_var fv = RCReader::readField(pos,octSeq);
     printField(out,fv);
     out << "|";
   }
   out << '\n';
 }
}
\end{verbatim}

\section{ Data Types }

\subsection{ FieldValue }

 This datatype represents one field in database. It defined in IDL via next
definitions:

\begin{verbatim} 
 ///
 typedef boolean Null;

 /**
  * this union represent one field in DB
  **/
 union FieldValue switch(Null){
        case FALSE : Value v;
 };
\end{verbatim} 


   I.e. FieldValue can be NIL (when aprropriative valis is NIL value in RDB),
    or can have value of type \verb|CosQueryCollection::Value|

  Now, let's look on \verb|Value| definitions:

\begin{verbatim}
  /**
   * what can be not null value in DB:
   **/
  union Value switch(FieldType) {
      ///
      case TypeBoolean: boolean b;
      ///
      case TypeChar: char c;
      ///
      case TypeOctet: octet o;
      ///
      case TypeShort : short s;
      ///
      case TypeUShort : unsigned short us;
      ///
      case TypeLong : long l;
      ///
      case TypeULong : unsigned long ul;
      ///
      case TypeFloat : float f;
      ///
      case TypeDouble : double d;
      ///
      case TypeString : string str;
      ///
      case TypeObject : Object obj;
      ///
      case TypeSmallInt : short si;
      ///
      case TypeInteger : long i;
      ///
      case TypeReal : float r;
      ///
      case TypeDoublePrecision : double dp;
      ///
      case TypeCharacter : string ch;
      ///
      case TypeDecimal : Decimal dec;
      ///
      case TypeNumeric : Decimal n;
      ///
      case TypeDateTime   : DateTime dt;
      ///
      case TypeRaw     : sequence<octet> raw;
      ///
      case TypeLongRaw     : sequence<octet> lrawid;
      ///
      case TypeLongString  : sequence<octet> lstrid;
      /// 
      case TypeWString  : string wstr;
      ///
      case TypeLongWString  : sequence<octet> lwstrid;
  };
\end{verbatim}
 
  Relations between SQL types and CORBA types are shown in next table:

\begin{footnotesize}
  $$
   \begin{array}{|c|c|c|c|}  \hline
                 &  CORBA  & SQL\_ORACLE                        & SQL\_InterBase \\ \hline
     TypeBoolean & boolean & CHAR(1) [in]                       & CHAR(1) [in] \\
     TypeChar    & char    & CHAR(1)                            & CHAR(1) \\
     TypeOctet   & octet   & CHAR(1) [in]                       & CHAR(1) [in]  \\
     TypeShort   & short   & NUMBER(5,0)                        & SMALLINT  \\
     TypeUShort  & unsigned short  & NUMBER(5,0)                & SMALLINT\\
     TypeLong    & long    & NUMBER(10,0)                       & INTEGER  \\
     TypeULong   & unsigned long & NUMBER(10,0)                 & INTEGER \\
     TypeFloat   & float  &  NUMBER(x,y) \;\; x < 5 \land y > 0 & FLOAT \\
     TypeDouble  & double &  NUMBER(x,y) \;\; x>5 \land y > 0   & DOUBLE PRECISION \\
     TypeString  & string & VARCHAR2(X)                       & VARCHAR(X) \\
     TypeObject  & Object & VARCHAR2(X) \lor UDF              & VARCHAR(X) \\
     TypeSmallInt & short & NUMBER(5,0)                       & SMALLINT \\
     TypeInteger  & long & NUMBER(10,0)                       & INTEGER \\
     TypeReal,   &  &                                        & FLOAT \\
     TypeDoublePrecision &  &                                & DOUBLE PRECISION \\
     TypeCharacter &  &                                       & CHAR(1) [in] \\
     TypeDecimal & CosQueryCollection::Decimal & NUMBER(x,y) \land x > 10   & NUMERIC \\
     TypeNumeric & CosQueryCollection::Decimal &                           & NUMERIC \\
     TypeDateTime & CosQueryCollection::DateTime & DATE & DATE\\
     TypeRaw  &  sequence<octet>  &  LOB &\\
     TypeLongRaw & sequnece<octet> & BLOB &\\
     TypeLongString & sequence<char> &  CLOB &\\
     TypeLongRaw & sequnece<octet> & BLOB                    & BLOB \\
     TypeLongString & sequence<octet> && \\
     TypeWString & wstring  & NCHAR, NVARCHAR &\\
     TypeLongWString & sequence<octet> && \\
     TypeBlob & CosQueryCollection::Blob & BLOB & BLOB \\
     TypeClob & CosQueryCollection::Clob & CLOB & BLOB \\
     TypeWclob & CosQueryCollection::Wclob & NCLOB & BLOB \\ \hline
   \end{array}
  $$
\end{footnotesize}
   
\subsubsection{ C++ Facade class  FieldValueAccess }

   You can directly work with FieldValue using IDL to C++ mapping, but it 
   is easier to use 
   access operators from class FieldValueAccess.
  \begin{itemize}
   \item \verb|FieldValueAccess::isNil(const FieldValue& fv)| returns 
    \verb|ithrue|, 
    if fv is NUL, \verb|false| otherwise
\begin{verbatim}
  if (FieldValueAccess::isNil(myRecord[0])) {
    cout << "NIL here" << endl;
  }
\end{verbatim}
   \item \verb|FieldValueAccess::setNil(FieldValue& fv)| set \verb|fv|fv to NIL. 
   \item For each field datatype XXX exists function FieldValueAccess::setXXX 
  and
   FieldValueAccess::getXXX for writing/reading of FieldValue.
 \newline
   for examle:
\begin{verbatim}
  CORBA::Long x = FieldValueAccess::getLong(fv); // read Long from fv
  FieldValueAccess::setLong(fv, 1); // set fv to 1.
\end{verbatim}
   If during reading fv, fv is NIL, then RC::FieldVualueIsNull throwed,
   if we try to read value of wrong type, then InvalidFieldTypeException is
   throwed.
   \item FieldValueAccess::setAsString and getAsString 
         allows to work with string representations of FieldValue.
   \item At least, class FieldValueAccess is defined in \verb|$(prefix)/include/CosQueryFacade|, so you must include this file before using FieldValueAccess.
\begin{verbatim}
#include <CosQueryFacade/FieldValueAccess.h>
\end{verbatim}
  \end{itemize}

\subsection{ Record, RecordSeq }
 Record type is just a sequence of field values and denotes type for one 
  record of DB. RecordSeq is a sequence of records.
 \newline
 IDL definitions:
\begin{verbatim}
 typedef sequence<FieldValue> Record;
 typedef sequence<Record> RecordSeq;
\end{verbatim}
 
The next code fragment shows how to create record from 2 fields: 
\verb|(2,'qqq')| of type \verb|(Short, String)|:
\begin{verbatim}
  Record_var record = new Record;
  record->length(2);
  FieldValueAccess::setShort(record[0],2);
  FieldValueAccess::setString(record[1],"qqq");
\end{verbatim}

\subsubsection{ C++ Facade classes RecordAccess, SRecord } 

  Appropriative Facade class RecordAccess defines
  \begin{itemize} 
    \item constant - empty record 
          (RecordAccess::empty()), 
    \item constant - empty sequence of records 

  (RecordAccess::emptySeq()), 
    \item constant - empty binary sequence (RecordAccess::emptyOctSeq())
    \item static method,
  which transform record to records sequence with one element.
    (RecordAccess::createSeq)
    \item family of addXxx methods
  \end{itemize} 

  Yet one Facade class is \verb|SRecord|, which provide ``syntax sugar''
  for records creation.

  Let we want to create record with 3 fields:
    \verb|(3 Short, "aa" String, "bb" String)|.

  Using \verb|FieldValueAccess| code fragment, for creating of this record
 will be looked like:
 \begin{verbatim}
   Record_var record = new Record();
   record->length(3);
   FieldValueAccess::setShort(record[0],3);
   FieldValueAccess::setString(record[1],"aa");
   FieldValueAccess::setString(record[2],"bb");
 \end{verbatim}

 \verb|SRecord| provide more comfortable API, which allows to create our record by one string of code:
\begin{verbatim}
  SRecord sr;
  sr._short(3)._string("aa")._string("bb");
\end{verbatim}

 i. e. for all database types \verb|Xxx| exists method  \verb|Srecord::_xxx|,
 which appendes field of such type to \verb|(*this)| and return one.

\subsection{ OctSeq, RCHeader }

In all interfaces UAKGQuery programmer has the choice: to pass data as set of notes or as binary set in RC-coding.
 
 Why use OctSeq -- because UAKGQuery defines own coding of database records to
 binary stream, which much more effective, than GIOP. With UAKGQuery it is
 possible to achieve performance of data transfer near theoretical
 maximum.

OctSeq defined in IDL as sequence of bytes:
\begin{verbatim}
  typedef sequence<octet> OctSeq;
\end{verbatim}


 You can write data to octet stream with help of RCWriter and read
 with help of RCReader. 
 \footnote{RC means Record Coding. }

 About structure of RC byte stream: it constists of header and data section.
 Full BNF specifications of RC-coding is situated in App3 \ref{RC-BNF}.
 You need this specifications only in case, when you want to read/write
 RC-coded sequence outside UAKGQuery access (for example, by python program).

UAKGQuery gives to the application programmer the RCWriter or RCReader class correspondingly for reading/recording of RC-sequences.

 UAKGqueryService provides helper classes, which incapsulate low-level operations, it is necessary to know only that RC-sequence constists of header and data;
 header is defined next way:
\begin{verbatim}
struct RCHeader
{
  octet version;          // protocol version
  long  nRecords;         // number of records in stream
  unsigned long nFields;  // number of fields in one record.
};
\end{verbatim}

\subsubsection{ RCReader }

RCReader is the class, whith number of static methods for reading-recording RC-streams.

 So, for reading of RC-coded stream we must first read
 header, define number of records in stream and then read this
 records.

 Now, let's return to function print RC, which reads RC stream:
 
\begin{verbatim}
 RCHeader header;
 CORBA::ULong pos=0;
 RCReader::readHeader(header,pos,octSeq);
 out << "Header: nRecords=" << header.nRecords << endl;
 out << "Header: nFields=" << header.nFields << endl;
\end{verbatim}

 As we can see, RCReader::readHeader(header,pos,octSeq) fill header
 and set \verb|pos| to offset of first record in octet stream.

\begin{verbatim}
 for(CORBA::Long nRecord=0; nRecord<header.nRecords; ++nRecord){
   for(CORBA::ULong nField=0; nField<header.nFields; ++nField){
     FieldValue_var fv = RCReader::readField(pos,octSeq);
     printField(out,fv);
     out << "|";
   }
   out << endl;
 }
\end{verbatim}

 RCReader::readField read one field and set \verb|pos| to offset of
next record.

 You can read all record by one call, i. e. previous code fragment
 can be rewritten next way:
 
\begin{verbatim}
 for(CORBA::Long nRecord=0; nRecord<header.nRecords; ++nRecord){
   Record_var record = RCReader::readRecord(pos,octSeq);
   printRecord(out,fv);
 }
\end{verbatim}

 Or just sequence of records in one call:

\begin{verbatim}
   RecordSeq_var recordSeq = RCReader::readRecordSeq(pos,octSeq);
\end{verbatim}

 Full description of RCReader you can find in API Reference .

\subsubsection{ RCWriter }

 RCWriter expose API for writing into RC stream. Next code fragment
 creates RC stream and writes one record to it:

\begin{verbatim}
  OctSeq_var octSeq = new OctSeq;
  CORBA::ULong pos;
  RCWriter::writeHeader(1,record.length(),pos,octSeq);
  RCWriter::writeRecord(record,pos,octSeq);
\end{verbatim}

 Of course, you can also write by field or by sequence of records.

 Note, that write the number of records to header you can after writing
of data. I.e. the next code will work:

\begin{verbatim}
  OctSeq_var octSeq = new OctSeq;
  CORBA::ULong pos;
  RCWriter::writeHeader(1,record.length(),pos,octSeq);
  RCWriter::writeRecord(record,pos,octSeq);
  RCWriter::writeRecordSeq(recordSeq,pos,octSeq);
  RCWriter::writeNRecords(recordSeq.length()+1);
\end{verbatim}


\subsection{ RecordDescription }

 RecordDescription is a description of database record. It is defined via next
IDL definitions:

\begin{verbatim}

   /**
    * struct for description of field size.
    * name: name of field in DB.
    * ValueType: field type.
    * size: size of field in bytes. (for strings: include \0, i. e. 
    *       for VARCHAR(x) size is x+1
    * precision (have sense only for NUMERIC types) - precision.
    * scale (have sense only for NUMERIC types) - scale, as signed byte.
    **/
   struct FieldDescription{
      string         name;  // name of field
      CosQueryCollection::FieldType      type; // type
      unsigned long  size; // size of field in bytes
      unsigned short precision; // precision (have sense only for NUMBER)
      short          scale; // 
   };
   typedef sequence<FieldDescription>  RecordDescription;

\end{verbatim}

 I.e. if we have RecordDescription, than we know names and types
 of Data Fields.

 Of course, exists Facade class, for easy access to RecordDescription.
 It's name is RecordDescriptionAccess.

\section{ Evaluation of queries }

 So, as you know, for evaluation of SQL queries you can use family of
methods \verb|QueryManager::evaluate_xxx|.

\subsection{ Queries without parameters  }

 Let's return to our first example of query evaluation:

\begin{verbatim}

 try {

   OctSeq_var octSeq = queryManager->evaluate_rc("select * from tab","SQL92",
                                    RecordDescriptionAccess::empty(),
                                    RecordAccess::emptyOctSeq()); 
 }catch(QueryInvalid ex){
   cerr << "QueryTypeInvalid:" << ex.why << endl;
 }catch(QueryTypeInvalid ex){
   cerr << "QueryTypeInvalid" << endl;
 }catch(QueryProcessingError ex){
   cerr << "QueryProcessingError:" << ex.why << endl;
 }

\end{verbatim}

 Now we know, that:
\begin{itemize}
  \item Return value is RC-coded octet sequence.
  \item first parameter - text of SQL query.
  \item second parameter - flags. 
  \item third parameter - type of query input parameters. In this case we
 pass empty RecordDescription, because this query has no parameters.
  \item  four - query parameters. In our case: it is empty
\end{itemize}

 Note, that many queries without input parameters exist in typical application, so we define special methods: \verb|evaluate_rc_e| and \verb|evaluate_records_e| fo such case.

 I.e. previous example we can rewrite in more compact from:

\begin{verbatim}

 try {

  OctSeq_var octSeq = queryManager->evaluate_rc_e("select * from tab","SQL92");

 }catch(QueryInvalid ex){
   cerr << "QueryTypeInvalid:" << ex.why << endl;
 }catch(QueryTypeInvalid ex){
   cerr << "QueryTypeInvalid" << endl;
 }catch(QueryProcessingError ex){
   cerr << "QueryProcessingError:" << ex.why << endl;
 }

\end{verbatrim}

 Next question: how differs \verb|evaluate_records| family of methods 
 from \verb|evaluate_rc| -- very simple, 
 \verb|evaluate_records| family of methods
  accept query parameters 
 and return query results as sequence of records instead RC-coded sequences.

\begin{verbatim}
 RecordSeq_var records;
 try {
  records = queryManager->evaluate_records_e("select * from tab","SQL92");
 }catch(QueryInvalid ex){
   cerr << "QueryTypeInvalid:" << ex.why << endl;
 }catch(QueryTypeInvalid ex){
   cerr << "QueryTypeInvalid" << endl;
 }catch(QueryProcessingError ex){
   cerr << "QueryProcessingError:" << ex.why << endl;
 \end{verbatim}
 

\subsection{ queries with parameters  }

 Now, let's look at third and fourth parameter of QueryEvaluator::evaluate.
 Sense of third parameter: RecordDescription - is description of record,
 which passed to query as parameter.
 \footnote{
 We suppose, reader is familiar with concepts of binding variables (or,
  in some terminology: host variables). If not, please look at
 \cite{ORACLE-1}
}

 4-th parameter - is a record (or records) with values of query
 parameters.

 Let's illustrate this by next example:
 query is
\begin{verbatim}
select name from empties where id=:ID
\end{verbatim}

 Where \verb|:ID| -- query parameter with type long.

 Example of evaluating such query is below:

\begin{verbatim}
char* getEmployeeNameById(long id)
{
  RecordDescription paramsDescription;
  RecordDescriptionAccess::appendLong(paramsDescription, "ID");
  Record_var params = new Record;
  params->length(1);
  FieldValueAccess::setLong(params[0],id);
  RecordSeq_var retval = queryManager->evaluate_record(
                                "select name from empls where id=:ID","",
                                paramsDescription, params);
  if (retval->length()==0) {
      throw IncorrectEmployeeId(id);
  }
  return FieldValueAccess::createString(retval[0][0]);
}

\end{verbatim}

 Note, that host variables can be used not only  for passing information
to database but also for retrieving data; for example, with the help of 
SQL construction \verb|select into|

 The family of methods \verb|evaluate_*_inout| is intendent for this purpose.

 I.e. after next call: 
\begin{verbatim}
 queryManager->evaluate_record_inout(
         "select name into :name from emps where id=:id",
         "SQL_92",
          recordDescription,
          record
  ) 
\end{verbatim}

 value of \verb|name| will be fetched into record.

\section{ Query Interface }

 Using of methods from family \verb|QueryEvaluator::evaluate| does not
overlap all needed functionality of database interface. 
 We come to problems with next cases:
 \begin{enumerate}
   \item We can't query large data sets by parts: all data are passed to client
  during one request.
   \item We can't query descriptions of received data set.
   \item SQL server fully parse query during each call of evaluate.
 \end{enumerate}

 So, we need more complex interface to use  all functionality of 
 relational database. This interface is named \verb|Query|

\begin{verbatim}

   interface Query
   {

     ///
     readonly attribute QueryManager query_mgr;

     /**
      *@return text of query.
      */
     readonly attribute string  queryText;
 

     /**
      * return status of query: i.e: 
      *  complete when query is executed, otherwise incomplete
      */
     CosQuery::QueryStatus get_status ();


     /**
      * prepare query for executing.
      * if query have no parameters, paramsDescription must be empty 
      * sequence.
      *@param paramsDescription  description of query input parameters.
      */
     void prepare_query(in RecordDescription paramsDescription)
                   raises(CosQuery::QueryProcessingError);

     /**
      * synonim for prepare_query 
      **/
     void prepare(in RecordDescription paramsDescription)
                   raises(CosQuery::QueryProcessingError);


     /**
      * execute query
      *@param octSeq_  RC-coded sequence of input parameters.
       *      can be empty, if query have no parameters.
      **/
     void execute_rc(in OctSeq octSeq_)
                            raises(CosQuery::QueryProcessingError);

     /**
      * execute query, and if query has out or inout parameter, then fill
      * them
      *@param octSeq_  RC-coded sequence of parameters.
      * parameter can have in, out and inout modes.
      **/
     void execute_rc_inout(inout OctSeq octSeq_)
                            raises(CosQuery::QueryProcessingError);
     /// 
     void execute_records(in RC::RecordSeq recordSeq_)
                            raises(CosQuery::QueryProcessingError);
     ///
     void execute_record(in CosQueryCollection::Record record_)
                            raises(CosQuery::QueryProcessingError);
     ///
     void execute_records_inout(inout RC::RecordSeq recordSeq_)
                            raises(CosQuery::QueryProcessingError);
     
     ///
     RecordDescription  get_result_description() 
                               raises(CosQuery::QueryProcessingError, 
                                      QueryNotPrepared);
     ///
     RecordDescription  get_parameters_description()
                               raises(CosQuery::QueryProcessingError);

     ///
     RC::RecordSeq get_all_parameters_records() 
                               raises(CosQuery::QueryProcessingError);
     ///
     RC::RecordSeq get_parameters_records(in StringSeq neededFields)
                               raises(CosQuery::QueryProcessingError,
                                      InvalidParameterName);
     ///
     OctSeq get_all_parameters_rc() 
                               raises(CosQuery::QueryProcessingError);
     ///
     OctSeq get_parameters_rc(in StringSeq fieldNames)
                               raises(CosQuery::QueryProcessingError,
                                      InvalidParameterName);

     /**
      *@returns number of fetched rows.
      */
     unsigned long  get_row_count() 
                       raises(CosQuery::QueryProcessingError);
     
     /**
      * fetch query result in records.
      * @param numberOfRecords -- number of records to fetch.
      *        0 means, that we want to fetch all records.
      * @param more -- true, if status is incomplete (i.e. we can query
      * more results), otherwise false.
      * @returns fetched rows packed in RC coding to octet sequence.
      **/
     OctSeq  fetch_rc(in unsigned long numberOfRecords, out boolean more)
                       raises(CosQuery::QueryProcessingError);

     /**
      * synonim for fetch_rc.
      */
     OctSeq  get_result_rc(in unsigned long numberOfRecords)
                       raises(CosQuery::QueryProcessingError);


     /**
      * fetch query result in records.
      * @param numberOfRecords -- number of records to fetch.
      *        0 means, that we want to fetch all records.
      * @param more -- true, if status is incomplete (i.e. we can query
      * more results), otherwise false.
      * @returns fetched records.
      **/
     RC::RecordSeq fetch_records(in unsigned long numberOfRecords, 
                             out boolean more)
                       raises(CosQuery::QueryProcessingError);

     /**
      * synonim for fetch_records
      */
     RC::RecordSeq  get_result_records(in unsigned long numberOfRecords)
                       raises(CosQuery::QueryProcessingError);

     /**
      * skip N records without retrieving.
      *@returns actual number of skipped records.
      */
     unsigned   long  skip(in unsigned long numberOfRecords,
                           out boolean more)
                       raises(CosQuery::QueryProcessingError);

     /**
      *@return last error.
      * if Query is ok, code in error is 0.
      */
     QueryError  get_last_error();

     /**
      * destroy query, which not longer needed
      **/
     void        destroy();


 };
\end{verbatim}

 To use \verb|Query| interface, application programmer must perform
 next steps:


 \begin{enumerate}
   \item create Query using method \verb|QueryManager::createQuery|
   \item prepare query, using method \verb|Query::prepare| 
   This method has one parameter: description of query binding variables.
   \item execute query, using one of family of \verb|Query::execute| methods.
   \item receive data if necessary, using one of methods 
  \verb|Query::fetch_XX|
   \item after end of work with query, delete it using mehtod \verb|Query::destroy|
 \end{enumerate}

  For example, receiving name of employer from id of database record 
 using query interface can be expressed in next code fragment:
 

\begin{verbatim}
  Query_var query = queryManager->
  RecordSeq_var query = queryManager->create_query(
                                "select name from empls where id=:ID","");
  query->prepare(paramsDescription); 
  query->execute_record(params);
  RecordSeq_var retval = query->fetch_rc(0); 
  query->destroy();
\end{verbatim}


\subsection { Prepared Query }

 You can prepare query once and then execute it many times
with different parameters. This can be more effective then creating
 new query for each request  because parsing and passing of parameters
description are performed only once. \footnote{Note, that it is possible 
 only inside one transaction }

 Typical using of this technique can be illustrated by next code fragment:

\begin{verbatim}
class EmployeeManager
{
 Query id2nameQuery_;
  
 ....
 
public:

 char* getNameById();

};

void EmployeeManager::init()
{
 ....
 id2nameQuery_ = qm_->create_query("select name from empls where id=:id","");
 id2nameQuery_->prepare(paramsDescription);
 .....
}

char* EmployeeManager::getNameById(CORBA::Long id)
{
 Record params(1);
 FieldValueAccess::setLong(params[0],id);
 id2nameQuery_ ->execute_record(params); 
 .....
}

EmployeeManager::~EmployeeManager()
{
  .....
  id2nameQuery_->destroy();
  ....
}

\end{verbatim}

 I.e. we keep creating and initialization of query in section of
class initialization, destroying of query in destructor and execution
of query we put into function, which do real work.

\subsection { Fetching data via Iterator pattern }

 Another feature of query interface -- API for sequential receiving of
data sets by parses. (According to {\em Iterator} design pattern).

 Let's look on next example:

\begin{verbatim}
 CORBA::Boolean more=true;
 while(more)
 {
   OctSeq_var rc = fetch_rc(chunkSize,more);
   ......
   do something;
 }
\end{verbatim}

 i. e. \verb|fetch| return \verb|chunkSize| or less records and set 
 \verb|more| to false if we receive all records of query result set.

\subsection { Receiving of result description  }

 We can receive description of query output  use method
 \verb|Query::get_result_description()|;

 The next code fragment returns our description of table, set by user:

\begin{verbatim}
  char tablename[MAX_TNAME_LEN];
  cout << "enter table name:";
  cout.flash();
  cin.getline(tablename,MAX_TNAME_LEN); 
  ostrstream ostr;
  ostr << "select * from " << tablename;
  Query_var query = queryManager->create_query(ostr.str(),"");
  ostr.rdbuf()->freze(0);
  RecordDescription_var tableDescription = query->get_result_description(); 
\end{verbatim}


\section{ QueryManager interface }

 
\subsection { Sequences }
 
NAME \\ 
\verb|get_next_sequence_value| - return next value from sequence. \\

\noindent SYNOPSIS 
\begin{verbatim}
  long  get_next_sequence_value(in string sequence_name); 
\end{verbatim}
  
\noindent DESCRIPTION \\
The function \verb|get_next_sequence_value| get next value from  sequence \verb|sequence_name|. \\


\noindent RETURN VALUE \\
\verb|get_next_sequence_value| return next value of sequence \verb|sequence_name|. 
  
\subsection { Properties }
  
NAME \\
\verb|get_property| - return value of property. \\ 
  
\noindent SYNOPSIS 
\begin{verbatim}
 string get_property(in string property_name)   raises(PropertyNotDefined); 
\end{verbatim}
   
\noindent DESCRIPTION \\
The function \verb|get_property| return value set to property \verb|property_name| if such property defined and assigned to some value. 
Enabled properties : 
\begin{itemize}
 \item collection 
\end{itemize}
if property \verb|collection| set to "yes" - then this QueryManager support collections. \\
     
\noindent RETURN VALUE  \\
\verb|get_property|  return value of property if such exist and have a value, else function raise PropertyNotDefined exception. 


\section{ Large Objects (LOBs)  }

  Features of operating with large objects of database (LOB) consists in:
 \begin{itemize}
   \item three interfaces - Blob, Clob, Wclob
   \item three methods of Query interface (create\_blob, create\_clob, create\_wclob)
 \end{itemize}
  Blob, Clob, Wclob can be parameters of query. 
  Objects of these types can be obtained from string returned.

  To accomplish query to database, parameter of which is LOB, do not possible 
  to use 'evaluate' mechanism. Query is single which has methods to create LOB
  to write. All LOB obtained as a result of query are read-only.

  Blob, Clob and Wclob interfaces are analogous. They differ one from other
  by data types only.
 \begin{itemize}
   \item Blob is meant for operating with binary arrays, an unit of which is octet byte,
         and UAKGQuery::OctSeq as data type is used;
   \item Clob is meant for operating with character type data CORBA::String is used;
   \item Wclob - for unicode, CORBA::WString is used;
 \end{itemize}
  Let us consider LOB using Blob:
\begin{verbatim}
    interface Blob {
        ///
        unsigned long length() raises(CosQuery::QueryProcessingError);
        ///
        UAKGQuery::OctSeq fetch_chunk(in unsigned long chunkSize
                , out boolean more)
                raises(CosQuery::QueryProcessingError, ForWritingOnly);
        ///
        void put_chunk(in UAKGQuery::OctSeq data)
                raises(CosQuery::QueryProcessingError, ForReadingOnly);
    };
\end{verbatim}
  Method {\bf length} return the length of LOB in units correspondent,
  (octet byte number for Blob, number of characters for Clob,
  number of unicode characters for Wclob).
  All dimensions are represented in these units.
  Method {\bf fetch\_chunk} can be called for LOB, which has been obtained as a result
  of select-type query. It provides mechanism to obtaining of real LOB data.
  {\bf chunkSize} argument determine maximal required size of readable part of LOB. 
  At that argument {\bf more}=true signify there are data for reading still.
  Null {\bf chunkSize} value means LOB must be readed wholly.
  Method put\_chunk can be called for LOB having been created via one of Query's
  methods create\_<B|C|Wc>lob only. Parameter {\bf data} represents actual data
  to write.
  On calling {\bf fetch\_chunk} and {\bf put\_chunk} in erroneous context an exception
  corresponding (ForWritingOnly, ForReadingOnly) will be thrown, pointing that the
  current LOB is write-only or read-only.
  Query's methods 
\begin{verbatim}
     /**
      * request Blob for filling query parameters
      *@returns empty Blob for writing only.
      */
     CosQueryCollection::Blob create_blob()
             raises(CosQuery::QueryProcessingError);

     /**
      * request Clob for filling query parameters
      *@returns empty Clob for writing only.
      */
     CosQueryCollection::Clob create_clob()
             raises(CosQuery::QueryProcessingError);

     /**
      * request Wclob for filling query parameters
      *@returns empty Wclob for writing only.
      */
     CosQueryCollection::Wclob create_wclob()
             raises(CosQuery::QueryProcessingError);
\end{verbatim}
 returns LOBs of type corresponding to write and following binding
 together with Query's parameters.
 This is example of reading:
\begin{verbatim}
    CORBA::String_var sql = CORBA::string_dup("select b from blob_test");
    RecordSeq_var rs = new RecordSeq();
    CORBA::Boolean more = true;
    try {
      Query_var q = queryManager->create_query(sql, "");
      q->prepare(RecordDescriptionAccess::empty());
      q->execute_records(RecordAccess::emptyRecordSeq());
      rs = q->fetch_records(1, more);
      if (rs->length() > 0) {
          Blob_var bl;
          try {
              bl = FieldValueAccess::getBlob(rs[0][0]);
          } catch (const InvalidFieldTypeException& ) {
              cerr << "InvalidFieldTypeException" << endl;
              return;
          } catch (const FieldValueIsNull& ) {
              cerr << "FieldValueIsNull" << endl;
              return;
          }
          FILE* f = fopen("file_from_blob", "wb");
          OctSeq_var octSeq;
          long n = 0;
          for (more=true; more; n++){
              octSeq = bl->fetch_chunk(chunk_size, more);
              fwrite(octSeq->get_buffer(), 1, octSeq->length(), f);
          }
          fclose(f);
      } else {
          cerr << "In database no records" << endl;
      }
      q->destroy();
    } catch(const QueryTypeInvalid& ex) {
        cerr << "QueryTypeInvalid" << endl;
        return;
    } catch(const QueryInvalid& ex) {
        cerr << "QueryInvalid" << endl;
        return;
    } catch(const QueryProcessingError& ex) {
        cerr << "QueryProcessingError" << endl;
        cerr << ex.why << endl;
        return;
    }
\end{verbatim}
 And this is example of writing:
\begin{verbatim}
  UAKGQuery::OctSeq_var octSeq = new UAKGQuery::OctSeq();
  CORBA::ULong chunk_size;

  ...// fill octSeq and choose chunk_size

  RecordDescription_var recordDescription = new RecordDescription;
  recordDescription->length(1);
  recordDescription[0].name=CORBA::string_dup(":param");
  recordDescription[0].type=TypeBlob;

  RecordSeq_var recordSeq = new RecordSeq;
  recordSeq->length(1);
  recordSeq[0].length(1);
  try {
    Query_var q = queryManager->create_query(
            "insert into blob_test(b) values(:param)", "");
    q->prepare(recordDescription);
    Blob_var bl = q->create_blob();
    if (chunk_size == 0) {
    	bl->put_chunk(octSeq.in());
	} else {
        CORBA::ULong sentLength = 0;
        CORBA::ULong leftLength = 0;
        const CORBA::Octet* curr;
        while (sentLength < octSeq.length()) {
            curr = octSeq.get_buffer() + sentLength;
            leftLength = octSeq.length() - sentLength;
            if (leftLength >= chunk_size)) {
                bl->put_chunk(OctSeq(chunk_size, chunk_size, curr, 0));
                sentLength += chunk_size;
            } else {
                bl->put_chunk(OctSeq(leftLength, leftLength, curr, 0));
                sentLength = octSeq.length();
            }
        }
	}
    FieldValueAccess::setBlob(recordSeq[0][0], bl);
    q->execute_records(recordSeq);
    q->destroy();
  } catch(const QueryTypeInvalid& ex) {
    cerr << "QueryTypeInvalid" << endl;
    return;
  } catch(const QueryInvalid& ex) {
    cerr << "QueryInvalid" << endl;
    return;
  } catch(const QueryProcessingError& ex) {
    cerr << "QueryProcessingError" << endl;
    cerr << ex.why << endl;
    return;
  }
\end{verbatim}


\section{ Collections  }


% $Id: CollectionsProgrammingGuide_eng.tex,v 1.10 2001-07-05 13:15:55 yulp Exp $

\subsection{ Introduction }

 Query Collection expose high-level object model, intendent for 
work with data sets, organized as so-called collections.

 Using of Query Collections allow to reduce low-level SQL programming for
 accessing and modifying of data: all typical operations are implemented
 in collections objects.

\subsection{ General description  }

\subsubsection{ ``Phisycal sense'' of collection }

 What is collection inside - just a set of SQL sentences for reading, writing
and modification of data, which are evaluated during call of collection object methods.

 Example: if you want to select some data from one table, you use
some SQL sentence which have form:

\begin{verbatim}
 SELECT <select-part> FROM <from-part> WHERE <where-part>
\end{verbatim}

So this SQL sentence defines some data set, which can be described by
 parts of our SQL sentence.
Imagine now, that we want to retrieve from this data set all records, which
satisfy some condition.To do it we can use such methods from typical collection interfaces 
as \verb|retrieve_by_filter|; in SQL we must evaluate SQL sentence,
which looks as 
\begin{verbatim}
 SELECT <select-part> FROM <from-part> WHERE <where-part> AND <condition>
\end{verbatim}
 Method \verb|update_by_filter| (intendent to update data, which 
satisfy some condition ) causes evaluating of next SQL sentence:
\begin{verbatim}
 UPDATE <select-part> SET <set-part> WHERE <where-part> AND <filter>
\end{verbatim}
Note, that \verb|<set-part>| can be automatically 
deduced from \verb|<select-part>|.

I. e. during call of collection methods appropriate SQL sentences are
built and evaluated.

 So, what is the query collection itself: object, which simply
 keeps and evaluates set of SQL sentences for data access and
 modifying.


In more formal terms we can say, that
 UAKGCollection is defined by:
 \begin{itemize}
   \item Set of data fields (i. e. SELECT part of SQL sentence)
   \item Data Source (i. e. FROM part of SQL sentence)
   \item Query condition (i. e. WHERE part of SQL sentence)
   \item Ordering (i. e. ORDER-BY part of SQL sentence)
 \end{itemize}


\subsection{ Steps for using collections interfaces  }

 For work with query collections, application programmer must
perform the next steps:

 \begin{itemize}
   \item create Collection using appropriate methods of QueryManager.
  \item Using UAKGCollection you can query or modify data in it, or
    receive \verb|Iterator| for sequential access to data set.
  \item Before end of work it is necessary to clear server resources,
   associated with  UAKGCollection by call of \verb|UAKGCollection::destroy| 
 \end{itemize}

\subsection{ Creation of Query Collections }

 UAKGQuery supports two types of collections: 
 UAKGCollection and UAKGKeyCollection.

 UAKGCollection represents data set, UAKGKeyCollection - dataset with unique
keys.
 \newline
 \newline
 Exists 2 ways of creating UAKGCollection:
\begin{itemize}
   \item by SQL sentence
   \item by set of field names and conditions. (i. e. by parts)
 \end{itemize}

for example, let's look at next two code fragments:

\begin{verbatim}
 UAKGCollection_var collection = queryManager->create_collection_by_parts(
                                          "F1,F2", "UAKGTEST", "F1=1", "F2");
\end{verbatim}

or

\begin{verbatim}
 UAKGCollection_var collection = queryManager->create_collection_by_query(
                         "select F1,F2 from UAKGTEST where F1=1 order by F2");
\end{verbatim}

 This 2 fragments give us identical result: collection which consists 
 of fields F1 and F2 of records in UAKTEST, ordered by F2, where F1 is 1.
 
 Parameters of \verb|create_collection_by_fields| in first code fragment are:

 \begin{enumerate}
   \item set of fields: SELECT part of SQL sentence
   \item data source: FROM part.
   \item select condition: WHERE part.
   \item ordering (optional) : ORDER part.
 \end{enumerate}

\subsection{ Data Access }

\subsubsection{ Receiving data with the help of Iterator's interface }

 You can use \verb|Iterator| concepts to retrive data.
 The steps for work with iterator are described below:

 \begin{enumerate}
  \item receive \verb|Iterator|, by call of \verb|create_iterator| method of collection.
  \item Use \verb|Iterator| API for navigation across data set.
  \item destroy \verb|Iterator|
 \end{enumerate}

Example:

\begin{verbatim}
 UAKGIterator_var iterator_ = collection_->create_iterator();
 Boolean more = true;
 while( more )
 { 
   OctSeq_var octSeq_ = iterator_->fetch_rc(50, more);
   printRC( octSeq_ );
 }
 iterator->destroy();
\end{verbatim}

 Interface \verb|Iterator| expose next methods:

 \begin{itemize}
   \item \verb|fetch_rc( ULong n, Boolean& more )|
     Read \verb|n| records as byte stream in RC-coding
   \item \verb|fetch_records( ULong n, Boolean& more )|
     Read \verb|n| records and return it as RecordSequence.
   \item \verb|skip( ULong n, Boolean& more )|
     Skip \verb|n| records without retrieving of data.
 \end{itemize}

\subsubsection{  Retryiving data via collection methods }

 You can read data from UAKGCollection
 directly, using next collection methods:

\begin{itemize}
 \item \verb|retrieve_by_filter( const char* where_filter )|
 \item \verb|retrieve_by_pattern( const Record& pattern )|
\end{itemize}

Both of those methods return RC-coded byte stream, which contains records
from data set collection filtered by parameter.
First method filters data set before retrieving by logical expression in \verb|where_filter|, second -- by principle of pattern matching (see \ref{patterns} ).

Example:

\begin{verbatim}
 UAKGCollection_var collection_ = queryManager->create_collection_by_query(
                          "select F1,F2 from UAKGTEST where F1=1 order by F2");
 OctSeq_var octSeq_ = collection_->retrieve_by_filter("F2='test'");
 printRC( octSeq_ );
 collection_->destroy();
\end{verbatim}

 This code fragment will print all records of created collection, where
F2 is equal to string \verb|'test'| . (I. e. all records in result will
have form \verb|(1,'test')| ).


\subsubsection{ Pattern matching }
 \label{patterns}

 Concept of pattern matching is often used in UAKGQuery Service for filtering
 data sets: for example, method \verb|retrieve_by_pattern| retrieves records
 which  match pattern, passed as argument of this method. Which means:
 pattern matching concept is derived from QBE (Query By Example) concept --
 in result set we receive records, which are "the same" as pattern, exclude
 fields, which set in pattern as empty.

More formal:

 Pattern is a record (i. e. have type \verb|Record|). Fields of pattern are
used for filtering data sets. 
Pattern matching for collection is defined as following:
\begin{itemize}
 \item Structure of pattern record  must be equal to structure of records in queuing datatset (i.e. number of fields and types must be equal)
 \item Record \verb|r| match pattern \verb|p| $iff$
 \begin{enumerate}
   \item $\forall i \in 0\dots length(r): p[i]!=NIL \rightarrow p[i]\; match \;r[i]$
    (i. e. for all field indexes, appropriate fields must match if pattern 
 field is not NULL).
   \item For fields $p_i,r_i$, $p_i\; match \;r_i$ iff
    \begin{enumerate}
       \item $type(r_i)==string \rightarrow (r_i\;\;LIKE\;\;p_i)$ 
       \item $type(r_i)!=string \rightarrow (r_i\;\;==\;\;p_i)$ 
    \end{enumerate}
    I. e. 2 fields match, if they are identical for non-string types, or
    if  LIKE matching is met for string types.
 \end{enumerate}
\end{itemize}

Practical example:\\

Suppose we have table UAKGTEST with two fields: F1 of type NUMBER and F2 of type VARCHAR2.
Then, we created collection which works with records of this table.
Then, for retrieving all records from this collection with F1=5 programmer must perform next steps:
\begin{itemize}
 \item Form pattern as a 2 field record , where first field must be 5, second - NULL . (i. e. second field  will not participate in matching).
 \item get needed records as result for \verb|retrieve_by_pattern| with pattern as parameter.
\end{itemize}

\begin{verbatim}
Record_var pattern = new Record;
pattern->length(2);
FieldValueAccess::setLong(pattern[0],5);
FieldValueAccess::setNull(pattern[1]);
OctSeq_var octSeq = collection_->retrieve_by_pattern(pattern);
printRC(octSeq);
\end{verbatim}

\subsection{ adding, updating and deleting of data }

  Of course, except retrieving of data, you can add, modify or delete collection items.

 List of aprropriate methods:
 \begin{enumerate}
  \item Adding data:
     \begin{itemize}
       \item \verb|add_record( const Record& element )| -- add one record to collection
       \item \verb|add_records( const RecordSeq& elements )| -- add sequence of records to collection.
       \item \verb|add_rc( const OctSeq& rc )| -- add sequence of records, coded in RC stream. 
     \end{itemize}

Note, that during adding of records no checking of belonging added data to 
target collection is performed.
For example, let's look at next code fragment:
\begin{verbatim}
 UAKGCollection_var collection=qm->create_collection(
                                         "select * from emp where deptno=22"
                                                    );
 ULong n=collection->number_of_records();
 cout << "now number of records is:" << n << endl;
 RecordDescription_var rd = collection->get_record_description();
 Record_var record=CosQueryFacade::RecordAccess::createRecordByDescription(rd);
 CosQueryFacade::RecordAccess::setShortByName(record.inout(),"EMPNO",11,rd);
 CosQueryFacade::RecordAccess::setShortByName(record.inout(),"DEPTNO",11,rd);
 collection->add_record(record);
 n=collection->number_of_records();
 cout << "now number of records is:" << n << endl;
\end{verbatim} 
 After successfull execution of this fragment one item will be added to \verb|emp| table, but number of elements in collection will not change.

  \item Updating of data:
     \begin{itemize}
       \item \verb|update_by_pattern( const Record& newRecord, const Record& pattern )| -- set to \verb|newRecord| records, which matching pattern \verb|pattern|. For example, next code fragment:
\begin{verbatim}
  SRecord sr1, sr2;
  collection_->update_by_pattern(
                 sr1._short(1)._short(2).in(), sr2_._nil()._short(2)
                               )
\end{verbatim}
    
       \item \verb|update_by_filter( const Record& newRecord, const char* filter )| - set to \verb|newRecord| all records, which satisfy logical condition in \verb|filter|.

For example, next code fragment:
\begin{verbatim}
   collection->update_by_filter(sr._short(1)._short(2),"x2=2");
\end{verbatim}
    will set to \verb|(1,2)| all records in collection for which \verb|x2=2|
     \end{itemize}
  \item Deleting data:
     \begin{itemize}
       \item \verb|remove_record( const Record& record )| - remove records, which are equal to given record.
       \item \verb|remove_records_by_filter( const char* filter )| - remove records by filter.
       \item \verb|remove_records_by_pattern( const Record& pattern )| - remove records by pattern.
       \item \verb|remove_all_records()| - remove all records from collection.
     \end{itemize}
 \end{enumerate}

Example:

\begin{verbatim}
  UAKGCollection_var collection_ = queryManager->create_collection_by_query(
                           "select F1,F2 from UAKGTEST where F1=1 order by F2");
  Record_var inpRecord = new Record;
  inpRecord->length(2);
  FieldValueAccess::setLong( inpRecord[0], 5);
  FieldValueAccess::setString( inpRecord[1], "test" );
  collection_->add_record(inpRecord);
  FieldValueAccess::setNull( inpRecord[1] );
  OctSeq_var octSeq_ = collection_->retrieve_by_pattern(inpRecord);
  printRC( octSeq_ );
  collection_->remove_records_by_pattern( inpRecord );
  collection_->destroy();
\end{verbatim}

\subsection{ Collection queries }

 The shown up set of methods is not always  sufficient for comfortable access to data collection.
Sometimes we need to perform more complicated actions:
for example change set of requested fields or receive some data from
linked table.

 For this purpose family of \verb|evaluate_xxx| methods of \verb|UAKGColletion| is intendent. (i. e. collections implements interface QueryEvaluator).

 For example, next query:
\begin{verbatim}
 result = collection->evaluate_rc_e("select x1,x3 from @ where x1=x2");
\end{verbatim}
 will fetch in result fields \verb|x1| and \verb|x2| for records, in which \verb|x1=x3|.

Next code fragment:
\begin{verbatim}
 collection=queryManager->create_collection("select * from orders");
 result=collection->evaluate_rc( 
   "select @,CUSTOMERS.NAME from @,CUSTOMERS where CUSTOMER.ID=customer_id
                              );
\end{verbatim}
 will retrieve all fields of order table and names of customers from linked table.  

Next query:
\begin{verbatim}
 result=collection->evaluate_rc("select @ from @ where not @")
\end{verbatim}
 will invert collection (i. e. select all records which are situated at the
same data source, but not belong to original collection).

Now, lets define semantics of our extending of SQL by \verb|@| :
\newline
If collection is based on SQL expression in form:
\begin{verbatim}
  select <select-part> from <from-part> where <where-part>
\end{verbatim}
and we pass in \verb|evaluate_xxx|  expression:
\begin{verbatim}
  select @,<new-select-part>
   from @,<new-where-part> [where <new-where-part>]
\end{verbatim}
 than result expression will have form:
\begin{verbatim}
  select <select-part>,<new-select-part>
   from <from-part>,<new-from-part>
   where (<where-part>) AND (<new-where-part>)
\end{verbatim}

 Suppressed SQL sentences is handled in less or more equal way.
 
\subsubsection{ Limitations }

  \begin{itemize}
    \item Colection can evaluate only SELECT queries without \verb|HAVING_BY| and \verb|GROUP_BY| clauses. Full grammar of Collection SQL is listed in 
  \ref{Collection-select-grammar}
  \end{itemize}

\subsubsection{ List of methods }

\begin{itemize}
 \item 
   \begin{verbatim}
      evaluate_rc( const char* queryText, const char* queryFlags, 
                   const RecordDescription& recordDescription, 
                   const OctSeq& params)
   \end{verbatim}
 \item 
   \begin{verbatim}
     evaluate_rc_inout(const char* queryText, const char* queryFlags, 
           const RecordDescription& recordDescription, OctSeq& params)
   \end{verbatim}
 \item 
   \begin{verbatim}
     evaluate_record( const char* queryText, const char* queryFlags, 
           const RecordDescription& recordDescription, const Record& params )
   \end{verbatim}
 \item 
   \begin{verbatim}
    evaluate_records_inout(const char* queryText, const char* queryFlags, 
                           const RecordDescription& recordDescription, 
                           RecordSeq& params)
   \end{verbatim}
 \item 
   \begin{verbatim}
     evaluate_records(const char* queryText, const char* queryFlags, 
                      const RecordDescription& recordDescription, 
                      const RecordSeq& params)
   \end{verbatim}
\end{itemize}

\subsection{ SubCollection|}

 As can be deduced from name of sections:  you can query some subset of collection in new collection (so-called subcollection).

Application programmer must perform next steps to use subcollection technique:

\begin{enumerate}
  \item Creation of \verb|SubCollection|
     \begin{itemize}
       \item \verb|create_subcollection( const char* subquery )| 
       In this method \verb|subquery| is expression on the same language, as
       for \verb|evaluate_xxx| family of collection methods. Created collection
       is result of evaluating such query as collection. Note, that if 
        data-source (i.e. where-part of subquery) contains multiple tables,
        then resulting collection is read-only.
       \item \verb|create_subcollection_by_pattern( const Record& pattern )|
       The resulting collection is just subset of records of original collection, which match pattern \verb|pattern|.
     \end{itemize}
  \item Work with received subcollection, using UAKGCollection API
  \item delete \verb|SubCollection| using method destroy()
 \end{enumerate}

Example:

\begin{verbatim}
 UAKGCollection_var collection_ = queryManager->
              create_collection_by_fields("tname, tabtype", "tab", "1=1", "");
 UAKGIterator_var iterator = collection_->create_uakgiterator();
 Boolean more;
 RecordSeq_var recordSeq = iterator->fetch_records(0, more);
 FieldValueAccess::setNull(recordSeq[0][0]);
 iterator->destroy();
 UAKGCollection_var new_collection_ = collection_->
                                create_subcollection_by_pattern(recordSeq[0]);
 iterator = new_collection_->create_uakgiterator();
 recordSeq = iterator->fetch_records(0, more);
 printRecordSeq(cout,recordSeq.in());
 iterator->destroy();
 new_collection_->destroy();
 collection_->destroy();
\end{verbatim}


\subsection{ UAKGKeyCollection}

 UAKGKeyCollection is a specialization of UAKGCollection with property
``have primary key''. \footnote{i. e. key, which is unique for each record in collection }.

 Additional methods provide API for accessing, modifying and deleting elements
by keys.

\subsubsection{ Creation of UAKGKeyCollection }

 for creation of KeyCollection you can use 2 methods of UAKGQueryManager:
\begin{itemize}
  \item \verb|create_key_collection_by_parts| -- which is create key collection,by appropriative parts of SQL sentence and yet ine additional part: list of field names, of which key is consists.
Example:
\begin{verbatim}
 UAKGKeyCollection_var collection = queryManager->
   create_key_collection_by_parts("F1,F2","UAKGTEST","F1=1","F2","F1");
\end{verbatim}
 \item \verb|create_key_collection_by_query| -- which is create key collection by SQL query , with yet one our extension of SQL: WITH KEY clause.
\begin{verbatim}
 UAKGKeyCollection_var collection = queryManager->
   create_key_collection_by_query(
    "select F1,F2  from UAKGTEST where F1=1 order by F2  with key F1"
                                 );
\end{verbatim}
  \verb|WITH KEY| clause can be used only for creation of key collection and
SQL sentence with it must have following syntax:
\begin{verbatim}
  SELECT <selection> <table-expr> WITH KEY <selection>
\end{verbatim} 
\end{itemize}

 As in SQL92 key can be compound:
\begin{verbatim}
  UAKGKeyCollection_var collection=queryManager->
           create_key_collection_by_query(
                     "select * from X with key x1, x2");
\end{verbatim}

-these two parts of code are equivalent,  they create the collection UAKGTest, where F1 is the primary key.  As we can see, we have one more SQL- clause\verb|WITH KEY|.
It may be used only to get collection with keys, enclosure syntax with this clause is following:

\begin{verbatim}
 SELECT <selection> <table-expr> WITH KEY <selection>
\end{verbatim}

As in SQL92,  the sequence of margins can create the key too:
 \begin{verbatim}
 UAKGKeyCollection_var collection = queryManager->
      create_key_collection_by_query(
                 "select * from X with key x1,x2");
\end{verbatim}


 Note, that specification of correct keys is business of application 
programmer : UAKGQueryService use this information, but does not do
any checking for accordance with real structure of database.

\subsubsection{ Methods of UAKGKeyCollection }

 UAKGKeyCollection provide next methods, for retrieving, updating
 and deleting  elements by keys:

 \begin{itemize}
   \item \verb|retrieve_record_with_key(const Record& key)| - retrieve record with key \verb|key|.
   \item \verb|retrieve_records_with_keys(const OctSeq& keys)| - retrieve sequence of records with accordance of \verb|keys|
   \item \verb|update_record_with_key(const Record& newRecord, const Record& key)| -- set to \verb|newRecord| element with key \verb|key|.
   \item \verb|update_records_with_keys(const OctSeq& records)| -- update records, with the same keys, as appropriate records in argument.
   \item \verb|remove_records_with_keys(const OctSeq& keys)| -- remove records with keys.
 \end{itemize}

 Also 2 helper methods are provided by UAKGKeyCollection:

 \begin{itemize}
   \item \verb|get_key_description()| -- return description of collection key
   \item \verb|extract_keys(const OctSeq& records)| -- extract keys from sequence of records.
 \end{itemize}

Example:

\begin{verbatim}
  UAKGKeyCollection_var collection_ = queryManager->create_key_collection_by_query("select F1,F2 from UAKGTEST where F1=1 with key F2");
  Record_var inpRecord_ = new Record;
  inpRecord_->length(1);
  FieldValueAccess::setString(inpRecord_[0], "test");
  collection_->remove_record_with_key( inpRecord );
  collection_->destroy();
\end{verbatim}

\subsection{ Using of UAKGCollectionListener }

 It becomes common and useful technique to organize program coupling via 
 notifivation about important events in life of some service.
 On this purpose UAKGCollection provide
 such mechanism of "Listeners" - user can add to collection own implementation
 of \verb|UAKGCollectionListener| callback interface, which would be called
 during performing of collection actions. 

\begin{verbatim}
 interface UAKGCollectionListener
 {
   // called, when elements are added:
   void elements_added(in OctSeq elements);

   //called, when elements updated
   void elements_updated(in OctSeq prev_elements,
                         in OctSeq new_elements); 

   //called, when elements removed:
   void elements_removed(in OctSeq elements)

   //called, when all ellements in collection are removed
   void all_elements_removed();

   //called, when collection is destroyed
   void collection_destroyed();
 }
\end{verbatim}

 Application programmer can implement this interface and bind it with
 collection for receiving events via method:
\begin{verbatim}
 unsigned long UAKGCollection::add_listener(
                                 in UAKGCollectionListener listener,
                                 in unsigned short eventMask);
\end{verbatim}
 Method returns number, which identifiy passed listener from
collection side. This number can be used for unbinding listener from
collection with help of method:
\begin{verbatim}
 unsigned long UAKGCollection::remove_listener(in unsigned long listenerIndex);
\end{verbatim}

 At last, \verb|eventMask| is a bit mask of events, which listener want to receive.


 Using notification technique remember, that cost of collection data for 
notificating can be high, and that synchronized calls of callback functions
on each event are not scalable technique. If you have situation, when N service
clients must receive notifications, than better do not register N listeners but 
create additional element of infrastructure, which receive notification and 
send it to clients, using asynchronics techniques (for example, via CORBA Event Service).


\subsection{ Limitations of using UAKGQuery collections interfaces }

For present tim , one instance of collection can be used in only 
one instance of transaction at the same time.

(I. e. concurrent access to collection from different transactions is 
not safe).




\section{ Transactions  }

 Two models of transactions are implemented in UAKQueryService:
 \begin{enumerate}
   \item XA transactions, which use XA resource of underlaying database and
    XA monitor of ORBacus Transaction Service.
   \item Own transaction manager, which shown as resource to ORB OTS and perform,
 all operations using specific transaction API of underlaying database.
 \end{enumerate}

 XA transaction manager is intended to be used in case, when your applications work
 in XA environment. UAKGQuery transaction manager is more light and
 in some cases can be more effective.
 
 Now, let's describe using of this two transaction models.

 \subsubsection{XA transactions}.

  Using of XA transactions are initiated by setting next flags in
command options of service:
 \begin{itemize}
   \item \verb|ORACLE_XA=<xa-open-string>| - for Oracle
   \item \verb|INTERBASE_XA=<xa-open-string>| - for Interbase.
 \end{itemize}
  
 Where \verb|<xa-open-string>| is XA string, with parameters of
database connection.
 For detail description of  XA string, look at documentation of
you database:
 \begin{itemize}
  \item Oracle:
   \verb|http://technet.oracle.com/doc/server.815/a68003/01_app1x.htm#619504|
  \item Interbase
    InterBase Programming Guide.
 \end{itemize}

 During work with XA transactions you must meet next limitations:

 \begin{enumerate}
  \item As far as connections to database is performed by XA monitor, then
   only one global XA connection (i. e. login,password,db) can be
 used by one UAKGQuery Server. Parameters \verb|username|and \verb|password|
 of \verb|DBConnectionManager::createQueryManager| are ignored.
 So, if you want to achieve one-time work of few different connections
 to one databases in XA mode, than you must start few copies of 
 UAKGQueryService.
  \item You can't use DLL statements in XA application.
  \item You can't call UAKGQuery methods outside of transaction context.
  \item You must use \verb|tread_per_request| threading mode.
 \end{enumerate}


 \subsubsection{UAKG OTS Transactions}.

  UAKG OTS transaction mode is used by default.
 
 Unlike XA transactions, UAKGQ transactions does not
touch parameters of db connections. All, what you need is call methods of
UAKGQuery in transactional context. You can mix transactional and
non-transactional calls: in this case all non-transactional calls will
be mapped to short local transactions.

\subsection{Typical Usage Of Transactions}

 Next code fragment illustrate typical usage of OTS:

\begin{verbatim}

 CORBA::Object_var obj = orb->resolve_initial_references("TransactionCurrent");
 CosTransactions::Current_var current = 
                                 CosTransactions::Current::_narrow(obj);

 current->begin();
 try {

   // do something with db:
   queryManger->evaluate_query(query1,"SQL92",query1ParamsDescriptions,
                                              query1Params);
   queryManger->evaluate_query(query2,"SQL92",query2ParamsDescriptions,
                                              query2Params);

   current->commit();

 }catch(const QueryProcessingError& ex){

   current->rollback();

 }catch(...){
   cerr << "Fatal: unknown exception";
   current->rollback();
 }
\end{verbatim}

 I.e. typical usage of transaction: to achieve atomity of sequence of
operations: all operations in sequence will be successful or all will be complete rollbacked.

 In details, transaction mechanisms are described in:
 \cite{OMG-COS-OTS} ,  \cite{XOPEN-XA}.

\section{IDL definitions}

\subsection{CosQueryCollection}
\begin{verbatim}
#ifndef __COSQUERYCOLLECTION_IDL
#define __COSQUERYCOLLECTION_IDL

/*
* module CosQueryCpllection.
* writeln from specifications of OMG froup for CORBA Query Service.
* (C) Ruslan Shevchenko <Ruslan@Shevchenko.Kiev.UA>
* (C) GradSoft <info@gradsoft.com.ua>
* 1998, 1999, 2000, 2001
* $Id: ProgrammingGuide_eng.tex,v 1.26 2002-12-05 12:01:22 kav Exp $
*/ 

#ifndef __CosQueryIDLConfigV2_idl
#include <CosQueryIDLConfigV2.idl>
#endif

#ifdef HAVE_ORB_IDL
#include <orb.idl>
#endif

#pragma prefix "gradsoft.kiev.ua"

/**
 *
 **/
module UAKGQuery 
{
    ///
    /**
     * UAKGQuery sequence of octets
     **/
    typedef sequence<octet> OctSeq;

};


#include <CosQuery.idl>


#pragma prefix "omg.org"


/**
* module CosQueryCpllection.
* data definitions for CORBA Query Service
* writeln from specifications of OMG group .
* (C) Ruslan Shevchenko <Ruslan@Shevchenko.Kiev.UA>
* 1998, 1999, 2000
**/
module CosQueryCollection {

  ///
  exception ElementInvalid {};
  ///
  exception IteratorInvalid {};
  ///
  exception PositionInvalid {};

  /**
   * possible DB field types
   **/
  enum FieldType {
     ///
     TypeBoolean, 
     ///
     TypeChar, 
     ///
     TypeOctet, 
     ///
     TypeShort,
     ///
     TypeUShort, 
     ///
     TypeLong, 
     ///
     TypeULong, 
     ///
     TypeFloat, 
     ///
     TypeDouble,
     ///
     TypeString, 
     ///
     TypeObject, 
     ///
     TypeSmallInt, 
     ///
     TypeInteger,
     ///
     TypeReal, 
     ///
     TypeDoublePrecision, 
     ///
     TypeCharacter, 
     ///
     TypeDecimal,
     ///
     TypeNumeric,
     ///
     TypeDateTime,
     ///
     TypeRaw,
     ///
     TypeWString,
     ///
     TypeBlob,
     ///
     TypeClob,
     ///
     TypeWclob
  };
 
  /**
   * decimal field.
   **/
  struct Decimal {
    /**
     * precision of number.
     **/
    long precision; 
    /**
     * scale of number
     **/
    long scale; 
    /**
     * valus in BCD format.
     **/
    sequence<octet> value;
  };

  /**
   * type, corresponding to DATE field.
   * (all values are start from 1)
   **/
  struct DateTime {
    ///
    short year;
    ///
    octet month;
    ///
    octet day;
    ///
    octet hour;
    ///
    octet minute;
    ///
    octet second;
  };
    ///
    exception ForReadingOnly {};

    ///
    exception ForWritingOnly {};

    /**
     *
     **/
    interface Blob {
        ///
        unsigned long length() raises(CosQuery::QueryProcessingError);
        ///
        UAKGQuery::OctSeq fetch_chunk(in unsigned long chunkSize
                , out boolean more)
                raises(CosQuery::QueryProcessingError, ForWritingOnly);
        ///
        void put_chunk(in UAKGQuery::OctSeq data)
                raises(CosQuery::QueryProcessingError, ForReadingOnly);

    };

    /**
     *
     **/
    interface Clob {
        ///
        unsigned long length() raises(CosQuery::QueryProcessingError);
        ///
        string fetch_chunk(in unsigned long chunkSize, out boolean more)
                raises(CosQuery::QueryProcessingError
                , ForWritingOnly);
        ///
        void put_chunk(in string data) raises(CosQuery::QueryProcessingError
                , ForReadingOnly);

    };

    /**
     *
     **/
    interface Wclob {
        ///
        unsigned long length() raises(CosQuery::QueryProcessingError);
        ///
        wstring fetch_chunk(in unsigned long chunkSize, out boolean more)
                raises(CosQuery::QueryProcessingError
                , ForWritingOnly);
        ///
        void put_chunk(in wstring data) raises(CosQuery::QueryProcessingError
                , ForReadingOnly);

    };


  /**
   * what can be not null value in DB:
   **/
  union Value switch(FieldType) {
      ///
      case TypeBoolean: boolean b;
      ///
      case TypeChar: char c;
      ///
      case TypeOctet: octet o;
      ///
      case TypeShort : short s;
      ///
      case TypeUShort : unsigned short us;
      ///
      case TypeLong : long l;
      ///
      case TypeULong : unsigned long ul;
      ///
      case TypeFloat : float f;
      ///
      case TypeDouble : double d;
      ///
      case TypeString : string str;
      ///
      case TypeObject : Object obj;
      ///
      case TypeSmallInt : short si;
      ///
      case TypeInteger : long i;
      ///
      case TypeReal : float r;
      ///
      case TypeDoublePrecision : double dp;
      ///
      case TypeCharacter : string ch;
      ///
      case TypeDecimal : Decimal dec;
      ///
      case TypeNumeric : Decimal n;
      ///
      case TypeDateTime   : DateTime dt;
      ///
      case TypeRaw     : sequence<octet> raw;
      /// 
      case TypeWString  : wstring wstr;
      ///
      case TypeBlob     : Blob bl;
      ///
      case TypeClob  : Clob cl;
      ///
      case TypeWclob  : Wclob wcl;

  };
 
 ///
 typedef boolean Null;

 /**
  * this union represent one field in DB
  **/
 union FieldValue switch(Null){
        case FALSE : Value v;
 };

 /**
  * one record in DB
  **/
 typedef sequence<FieldValue> Record;

 /**
  *
  **/
 typedef string Istring;



};

#endif
\end{verbatim}
\subsection{CosQuery}
\begin{verbatim}
#ifndef __COSQUERY_IDL
#define __COSQUERY_IDL

/*
* module CosQuery.
* from specifications of OMG group for CORBA Query Service.
* (C) Ruslan Shevchenko <Ruslan@Shevchenko.Kiev.UA>, 1998, 1999, 2000
* (C) GradSoft, 2001
* $Id: ProgrammingGuide_eng.tex,v 1.26 2002-12-05 12:01:22 kav Exp $
*/ 

#include <CosQueryCollection.idl>

#pragma prefix "omg.org"

/**
 * CosQuery: legacy definitions from OMG Query Service.
 **/
module CosQuery {

      ///
      exception QueryInvalid 
      { 
       ///
       string why; 
      };

      ///
      exception QueryProcessingError 
      { 

       ///
       string why; 
      };

      ///
      exception QueryTypeInvalid { };

      /// 
      enum QueryStatus 
      {
       ///
       complete, 
       ///
       incomplete
      };

};

#endif
\end{verbatim}

\subsection{UAKGQuery}
\begin{verbatim}
#ifndef __UAKGQUERY_IDL
#define __UAKGQUERY_IDL

/*
* IDL Definitions for UAKGQueryService
* (C) Ruslan Shevchenko <Ruslan@Shevchenko.Kiev.UA>, 1998,1999, 2000, 2001
* (C) GradSOft 2001
* $Id: ProgrammingGuide_eng.tex,v 1.26 2002-12-05 12:01:22 kav Exp $
*/

#ifdef CORBA_HAVE_OTS
#ifndef __COSTRANSACTIONS_IDL
#include <CosTransactions.idl>
#endif
#endif

#ifndef __COSQUERY_IDL
#include <CosQuery.idl>
#endif

#ifndef __RC_IDL
#include <RC.idl>
#endif

#pragma prefix "gradsoft.kiev.ua"

/**
 * UAKGQuery module
 * (GradSoft-specific type of UAKGQuery implementation).
 **/
module UAKGQuery 
{

    ///
    typedef sequence<string> StringSeq;

   /**
    * struct for description of field size.
    * name: name of field in DB.
    * ValueType: field type.
    * size: size of field in bytes. (for strings: include \0, i. e. 
    *       for VARCHAR(x) size is x+1
    * precision (have sense only for NUMERIC types) - precision.
    * scale (have sense only for NUMERIC types) - scale, as signed byte.
    **/
   struct FieldDescription{
      /// name of field in db
      string         name;
      /// field type
      CosQueryCollection::FieldType      type;
      /// size of field in bytes (for strings: include trailing \0, i. e.
      /// for VARCHAR2(x) size is x+1 
      unsigned long  size;
      /// precision (have sense only for numeric types)
      unsigned short precision;
      /// scale (have sense only for numeric types)
      short          scale;
   };

   ///
   typedef sequence<FieldDescription>  RecordDescription;

  /*
   struct ParameterDescription
   {
     FieldDescription      field;
     CORBA::ParameterMode  mode;
   };
   typedef sequence<ParameterDescription>  ParametersDescription;
  */
     
   ///
   struct QueryError
   {
    /// error code: 0 is OK.
    long   errorCode;  
    /// error message
    string errorMessage;
    /// sql string, during execution of which error causes.
    string sqlString;
    /// db name
    string dbName;
    /// error code from underlaying database
    long dbErrorCode;
   };

   ///
   exception QueryNotPrepared {};
   ///
   exception InvalidParameterName{};
   ///
   exception InvalidParameterType{};
   ///
   exception PropertyNotDefined{};


   /**
    * Hight level interface for evaluationg SQL queries
    **/
   interface QueryEvaluator
#ifdef CORBA_HAVE_OTS
                         //   :CosTransactions::TransactionalObject
                         // in all ORB-s context is passing uncoditionally 
#endif
   {

     /**
      * evaluate query <code> queryText </code> and return result as
      * RC-coded octet sequence.
      *@param queryText -- text of query
      *@param queryFlags -- flags for query executing
      *@param recordDescription -- description of input parameters.
      *@param params -- input parameters as RC-coded octet sequence
      *@return result of query
      **/
     OctSeq evaluate_rc(in string queryText, in string queryFlags,
                        in RecordDescription recordDescription_,
                        in OctSeq params)
              raises(CosQuery::QueryTypeInvalid,
                     CosQuery::QueryInvalid,
                     CosQuery::QueryProcessingError);

     /**
      * evaluate query <code> queryText </code> and return result as
      * sequence of records.
      *@param queryText -- text of query
      *@param queryFlags -- flags for query executing
      *@param recordDescription -- description of input parameters.
      *@param params -- input parameters as record sequence.
      *@return result of query
      **/
     RC::RecordSeq evaluate_records(in string queryText, in string queryFlags,
                                in RecordDescription recordDescription_,
                                in RC::RecordSeq params)
              raises(CosQuery::QueryTypeInvalid,
                     CosQuery::QueryInvalid,
                     CosQuery::QueryProcessingError);



     /**
      * evaluate query <code> queryText </code> and return result as
      * RC-coded octet sequence.
      *@param queryText -- text of query
      *@param queryFlags -- flags for query executing
      *@param recordDescription_ -- description of input parameters.
      *@param params -- input parameters as record .
      *@return result of query
      **/
     RC::RecordSeq evaluate_record(in string queryText,
                                                in string queryFlags,
                                in RecordDescription recordDescription_,
                                in CosQueryCollection::Record params)
              raises(CosQuery::QueryTypeInvalid,
                     CosQuery::QueryInvalid,
                     CosQuery::QueryProcessingError);


     /**
      * evaluate query <code> queryText </code> without bind parameters
      * and return result as RC-coded octet sequence.
      *@param queryText -- text of query
      *@param queryFlags -- flags for query executing
      *@return result of query
      **/
     OctSeq evaluate_rc_e(in string queryText, in string queryFlags)
              raises(CosQuery::QueryTypeInvalid,
                     CosQuery::QueryInvalid,
                     CosQuery::QueryProcessingError);

     /**
      * evaluate query <code> queryText </code> without bind parameters
      * and return result as sequence of records.
      *@param queryText -- text of query
      *@param queryFlags -- flags for query executing
      *@return result of query
      **/
     RC::RecordSeq evaluate_records_e(in string queryText, in string queryFlags)
              raises(CosQuery::QueryTypeInvalid,
                     CosQuery::QueryInvalid,
                     CosQuery::QueryProcessingError);
     
     /**
      * evaluate query <code> queryText </code> and fill out and inout 
      * parameters of queury, return result as RC-coded octet sequence.
      *@param queryText -- text of query
      *@param queryFlags -- flags for query executing
      *@param recordDescription_ -- description of input parameters.
      *@param params -- input parameters as record .
      *@return result of query
      **/
     OctSeq evaluate_rc_inout(in string queryText, in string queryFlags,
                              in RecordDescription recordDescription_,
                              inout OctSeq params)
              raises(CosQuery::QueryTypeInvalid,
                     CosQuery::QueryInvalid,
                     CosQuery::QueryProcessingError);

     /**
      * evaluate query <code> queryText </code> and fill out and inout 
      * parameters of queury, return result as sequence of records.
      *@param queryText -- text of query
      *@param queryFlags -- flags for query executing
      *@param recordDescription_ -- description of input parameters.
      *@param params -- input parameters as record .
      *@return result of query
      **/
     RC::RecordSeq evaluate_records_inout(in string queryFlags, 
                              in string queryType,
                              in RecordDescription recordDescription_,
                              inout RC::RecordSeq params)
              raises(CosQuery::QueryTypeInvalid,
                     CosQuery::QueryInvalid,
                     CosQuery::QueryProcessingError);


   };



   interface Query;
   interface QueryManager;

   /**
    * this is interface for UAKG Query
    * Query is SQL text with set of parameters: prepare parameters and 
    * execute parameters.
    *  prepare parameters are descriptionas of appropriative execute parameters
    *  execute parameters are SQL host valiables.
    * i. e. let we have query (SELECT * from T where x=:x and y=:y);
    * than prepare query have type RecordDescription and consist from
    * FieldDescription of :x and :y.
    * execute query are values of :x and :y (or sequence of pair of values
    * for multiple evaluated query).
    */
   interface Query
   {

     /**
      *@return owner of query
      **/
     readonly attribute QueryManager query_mgr;

     /**
      *@return text of query.
      */
     readonly attribute string  queryText;
 

     /**
      * return status of query: i.e: 
      *  complete when query is executed, otherwise incomplete
      */
     CosQuery::QueryStatus get_status ();


     /**
      * prepare query for executing.
      * if query have no parameters, paramsDescription must be empty 
      * sequence.
      */
     void prepare_query(in RecordDescription paramsDescription)
                   raises(CosQuery::QueryProcessingError);

     /**
      * synonim for prepare_query 
      **/
     void prepare(in RecordDescription paramsDescription)
                   raises(CosQuery::QueryProcessingError);


     /**
      * execute query
      *@params octSeq_  records of execute parameters, coded as RCSeq
      * (note, that prepare parameters is record descriptio of execute
      *  record).
      **/
     void execute_rc(in OctSeq octSeq_)
                            raises(CosQuery::QueryProcessingError);

     /**
      * execute query with inout parameters
      *@params octSeq_  records of execute parameters, coded as RCSeq
      **/
     void execute_rc_inout(inout OctSeq octSeq_)
                            raises(CosQuery::QueryProcessingError);
     
     /**
      * execute query
      *@params records -- query host parameters in RecordSeq
      * (query will be evaluated records.length() times)
      **/
     void execute_records(in RC::RecordSeq records)
                            raises(CosQuery::QueryProcessingError);

     /**
      * execute query
      *@params record_ -- query host parameters in one recod
      **/
     void execute_record(in CosQueryCollection::Record record_)
                            raises(CosQuery::QueryProcessingError);

     ///
     void execute_records_inout(inout RC::RecordSeq recordSeq_)
                            raises(CosQuery::QueryProcessingError);
     
     ///
     RecordDescription  get_result_description() 
                               raises(CosQuery::QueryProcessingError, 
                                      QueryNotPrepared);

     /**
      * get description of records parameters
      *@precondition
      * must be called after prepare
      **/
     RecordDescription  get_parameters_description()
                               raises(CosQuery::QueryProcessingError);

     ///
     RC::RecordSeq get_all_parameters_records() 
                               raises(CosQuery::QueryProcessingError);

     ///
     RC::RecordSeq get_parameters_records(in StringSeq neededFields)
                               raises(CosQuery::QueryProcessingError,
                                      InvalidParameterName);

     ///
     OctSeq get_all_parameters_rc() 
                               raises(CosQuery::QueryProcessingError);

     ///
     OctSeq get_parameters_rc(in StringSeq fieldNames)
                               raises(CosQuery::QueryProcessingError,
                                      InvalidParameterName);

     /**
      *@returns number of fetched rows.
      */
     unsigned long  get_row_count()
                       raises(CosQuery::QueryProcessingError);

     /**
      * fetch query result in records.
      * @param numberOfRecords -- number of records to fetch.
      *        0 means, that we want to fetch all records.
      * @param more -- true, if status is incomplete (i.e. we can query
      * more results), otherwise false.
      * @returns fetched rows packed in RC coding to octet sequence.
      **/
     OctSeq  fetch_rc(in unsigned long numberOfRecords, out boolean more)
                       raises(CosQuery::QueryProcessingError);

     /**
      * synonim for fetch_rc.
      */
     OctSeq  get_result_rc(in unsigned long numberOfRecords)
                       raises(CosQuery::QueryProcessingError);


     /**
      * fetch query result in records.
      * @param numberOfRecords -- number of records to fetch.
      *        0 means, that we want to fetch all records.
      * @param more -- true, if status is incomplete (i.e. we can query
      * more results), otherwise false.
      * @returns fetched records.
      **/
     RC::RecordSeq fetch_records(in unsigned long numberOfRecords,
                             out boolean more)
                       raises(CosQuery::QueryProcessingError);

     /**
      * synonim for fetch_records
      */
     RC::RecordSeq  get_result_records(in unsigned long numberOfRecords)
                       raises(CosQuery::QueryProcessingError);

     /**
      * skip N records without retrieving.
      *@returns actual number of skipped records.
      */
     unsigned   long  skip(in unsigned long numberOfRecords,
                           out boolean more)
                       raises(CosQuery::QueryProcessingError);


     /**
      * request Blob for filling query parameters
      *@returns empty Blob for writing only.
      */
     CosQueryCollection::Blob create_blob()
             raises(CosQuery::QueryProcessingError);

     /**
      * request Clob for filling query parameters
      *@returns empty Clob for writing only.
      */
     CosQueryCollection::Clob create_clob()
             raises(CosQuery::QueryProcessingError);

     /**
      * request Wclob for filling query parameters
      *@returns empty Wclob for writing only.
      */
     CosQueryCollection::Wclob create_wclob()
             raises(CosQuery::QueryProcessingError);


     /**
      *@return last error.
      * if Query is ok, code in error is 0.
      */
     QueryError  get_last_error();

     /**
      * destroy query, which not longer needed
      **/
     void        destroy();

   };

//
// UAKGQueryCollections
//


 interface UAKGCollectionListener;
 interface UAKGIterator;

 ///
 exception ReadOnlyCollection {};
 ///
 exception ReadOnlyIterator {};
 ///
 exception KeyNotFound {};

  ///
 interface UAKGCollection: QueryEvaluator
 {

   ///
   readonly attribute string  selectQueryText;
   ///
   readonly attribute string  selectDistinctQueryText;
   ///
   readonly attribute string  selectRangeQueryText;
   ///
   readonly attribute string  countQueryText;
   ///
   readonly attribute string  insertQueryText;
   ///
   readonly attribute string  removeAllQueryText;
   ///
   readonly attribute string  orderByText;

   ///
   RecordDescription  getRecordDescription()
                          raises(CosQuery::QueryProcessingError);

   ///
   void     set_readonly(in boolean rdonly)
                          raises(ReadOnlyCollection);

   ///
   boolean  is_readonly();

   /**
    * true, is select collection is ordered.
    **/
   readonly   attribute boolean    sorted;

   /**
    * add record
    **/
   void       add_record(in CosQueryCollection::Record element) 
                   raises(CosQueryCollection::ElementInvalid,
                          CosQuery::QueryProcessingError,
                          ReadOnlyCollection);


   /**
    * add records
    **/
   void       add_records(in RC::RecordSeq elements) 
                   raises(CosQueryCollection::ElementInvalid,
                          CosQuery::QueryProcessingError,
                          ReadOnlyCollection);


   /**
    * add records coded in RC sequence
    **/
   void       add_rc(in OctSeq rc)
                       raises(CosQueryCollection::ElementInvalid,
                              CosQuery::QueryProcessingError,
                              ReadOnlyCollection);


     //
     // retrieve record number

   /**
    *return number of records in collection
    *@returns number of records in collection
    **/
   unsigned long  get_number_of_records()
                         raises(CosQuery::QueryProcessingError);

     //
     // retrieve records

   /**
    * retrieve records by filter. 
    *@param where-filter : logical expression for selection of records 
    *   to delete (in SQL-like DBs is context of where clause)
    *TODO: what it return is it correct ?
    */
   OctSeq     retrieve_by_filter(in string where_filter)
                         raises(CosQuery::QueryProcessingError);

   
   /**
    * retrieve records by pattern.
    *@param : pattern
    *TODO: what it return is it correct ?
    **/
   OctSeq     retrieve_by_pattern(in CosQueryCollection::Record pattern)
                         raises(CosQuery::QueryProcessingError,
                                CosQueryCollection::ElementInvalid);


     //
     // replacing
     //

   /**
    * update records by pattern
    *@param  newRecord -- new record instead pattern matched
    *@param  pattern -- pattern for matching 
    **/
   void       update_by_pattern(in CosQueryCollection::Record newRecord, 
                                in CosQueryCollection::Record pattern ) 
                       raises(CosQuery::QueryProcessingError,
                              CosQueryCollection::ElementInvalid,
                              ReadOnlyCollection);

   /**
    * update records by filter
    *@param  newRecord -- new record instead filter matched
    *@param  filter --  condition
    **/
   void       update_by_filter( in CosQueryCollection::Record newRecord, 
                                in string filter )
                       raises(CosQuery::QueryProcessingError,
                              CosQueryCollection::ElementInvalid,
                              ReadOnlyCollection);

     //
     // removing
     //

   /**
    * remove all records from collection
    **/
   void       remove_all_records()
                      raises(CosQuery::QueryProcessingError,
                             ReadOnlyCollection);

   /**
    * remove records with same value as <code> record_ </code>
    *@param record_ - value of record to be removed.
    **/
   void       remove_record(in CosQueryCollection::Record record_)
                       raises(CosQuery::QueryProcessingError,
                              CosQueryCollection::ElementInvalid,
                              ReadOnlyCollection);

   /**
    * remove records with are satisficated to <code> filter </code>
    *@param filter - logical expression for selectiong removed records.
    **/
   void      remove_records_by_filter(in string filter)
                       raises(CosQuery::QueryProcessingError,
                              ReadOnlyCollection);

   /**
    * remove records with are match pattern  <code> pattern </code>
    *@param pattern - pattern to match.
    **/
   void      remove_records_by_pattern(in CosQueryCollection::Record pattern)
                       raises(CosQuery::QueryProcessingError,
                              ReadOnlyCollection);

    //
    // elements ordering
    //

   /**
    * sort - set new order expression
    *@param order_expressinon - new expresiion for ORDER BY clause
    **/
   void       sort(in string order_expression)
                         raises(CosQuery::QueryProcessingError);


    //
    // access interfaces factories
    //
                              
   /**
    * create iterator
    **/ 
   UAKGIterator    create_iterator();

   /**
    * create iterator which iterate records, matched for pattern
    **/
   UAKGIterator    create_iterator_by_pattern(
                            in CosQueryCollection::Record pattern)
                       raises(CosQueryCollection::ElementInvalid,
                              CosQuery::QueryProcessingError);

    /**
     *
     * subquery must be specified in next form:
     * <code>
     *   select <field_list> from <table_list> 
     *       where <conditions> [order by <field_list>]
     * </code>
     */
   UAKGCollection  create_subcollection(in string subquery)
                raises(CosQuery::QueryInvalid, 
                       CosQuery::QueryProcessingError);

   ///
   UAKGCollection  create_subcollection_by_pattern(
                                     in CosQueryCollection::Record pattern)
                raises(CosQuery::QueryInvalid, 
                       CosQuery::QueryProcessingError,
                       CosQueryCollection::ElementInvalid);


   /**
    * add listener to collection events
    **/
   unsigned long  add_listener(in UAKGCollectionListener listener,
                               in unsigned short eventMask);

   /**
    * remove listener
    **/
   boolean   remove_listener(in unsigned long listenerIndex);

   /**
    * destroy collection and free server resources, associated with
    * this collection.
    **/
   void      destroy();
 };

 ///
 interface UAKGCollectionListener
 {
   ///
   void  elements_added(in OctSeq elements);
   ///
   void  elements_updated(in OctSeq prev_elements, 
                           in OctSeq new_elements);
   ///
   void  elements_removed(in OctSeq elements);
   ///
   void  all_elements_removed();
   ///
   void  collection_destroyed();
 };

 struct ListenersSeqStruct
 {
  UAKGCollectionListener listener;
  unsigned short         mask;
 };

 typedef sequence<ListenersSeqStruct> UAKGCollectionListeners;
 
 /**
  * Iterator for retrieving data
  **/
 interface UAKGIterator
 {
    /**
     * are we situated at the end of data set ?
     **/
    readonly attribute boolean  end;
    
    /**
     * fetch n records as RC-coded octet sequence
     *@param n - number of records to fetch
     *@param more - set to true, if we not at end of collection.
     *@returns fetched records.
     */
    OctSeq      fetch_rc(in unsigned long n, out boolean more);

    /**
     * fetch n records as records sequence
     *@param n - number of records to fetch
     *@param more - set to true, if we not at the end of collection.
     *@returns fetched records.
     */
    RC::RecordSeq   fetch_records(in unsigned long n, out boolean more);

    /**
     * skip n records 
     *@param n - number of records to skip
     *@param more - set to true, if we not at the end of collection.
     *@returns actual number of skipped records.
     */
    unsigned long  skip(in unsigned long n, out boolean more);

    /**
     * destroy iterator and free associated server resources.
     */
    void	destroy();
 };

 /**
  * Collection of records with unique keys.
  */
 interface UAKGKeyCollection: UAKGCollection
 {

    ///
    RecordDescription  get_key_description();

    ///
    CosQueryCollection::Record  
              retrieve_record_with_key(in CosQueryCollection::Record key)
                         raises(CosQuery::QueryProcessingError);
                                

    ///
    void update_record_with_key(in CosQueryCollection::Record newRecord,
                                in CosQueryCollection::Record key)
                         raises(CosQuery::QueryProcessingError, KeyNotFound);

    ///
    void remove_record_with_key(in CosQueryCollection::Record key)
                         raises(CosQuery::QueryProcessingError);


    ///
    OctSeq retrieve_records_with_keys(in OctSeq keys)
                         raises(CosQuery::QueryProcessingError);

    ///
    void update_records_with_keys(in OctSeq records)
                         raises(CosQuery::QueryProcessingError);

    ///
    void remove_records_with_keys(in OctSeq keys)
                         raises(CosQuery::QueryProcessingError);

 };

 /**
  * factory for collection interfaces
  **/
 interface UAKGCollectionFactory
 {


    /**
     * queryText - select <field_list> from <table_list>
     * where <conditions> [order by <field_list>]
     */
    UAKGCollection create_collection( in string queryText )
                                      raises(CosQuery::QueryInvalid,
                                             CosQuery::QueryProcessingError);

    /**
     * queryText - select <field_list> from <table_list>
     * where <conditions> [order by <field_list>] with key [key-fields-list]
     */
    UAKGKeyCollection create_key_collection(
                                     in string queryText
                                           )
                                      raises(CosQuery::QueryInvalid,
                                             CosQuery::QueryProcessingError);

    ///
    UAKGCollection create_collection_by_parts(
                                      in string selectPartText,
                                      in string fromPartText,
                                      in string wherePartText,
                                      in string orderByPartText)
                                   raises(CosQuery::QueryInvalid,
                                             CosQuery::QueryProcessingError);

    ///
    UAKGKeyCollection create_key_collection_by_parts(
                                      in string selectPartText,
                                      in string fromPartText,
                                      in string wherePartText,
                                      in string orderByPartText,
                                      in string keysPartText)
                                   raises(CosQuery::QueryInvalid,
                                          CosQuery::QueryProcessingError);

 };

   /**
    * interface for our QueryManager.
    */
   interface QueryManager: QueryEvaluator,  
                           UAKGCollectionFactory
   {
      ///
      string get_username()  raises(CosQuery::QueryProcessingError);
      ///
      string get_dblink()  raises(CosQuery::QueryProcessingError);

      ///
      readonly attribute unsigned long number_of_queries;

      ///
      Query create_query(in string query, in string flags)
                         raises(CosQuery::QueryTypeInvalid,
                                CosQuery::QueryInvalid);

      ///
      Query create(in string query, in string flags)
                         raises(CosQuery::QueryTypeInvalid,
                                CosQuery::QueryInvalid);

      ///
      string get_property(in string property_name)
                          raises(PropertyNotDefined);

      ///
      long  get_next_sequence_value(in string sequence_name)
                        raises(CosQuery::QueryProcessingError,
			       CosQuery::QueryInvalid,
			       CosQuery::QueryTypeInvalid);
      
      ///
      void destroy();

   };

    ///
    exception QueryManagerNotFound {};
    
    typedef sequence<QueryManager>  UAKGQueryManagerSeq;

    ///
    interface DBConnectionManager
    {
     ///
     QueryManager createQueryManager(in string login, in string password, 
                           in string db_name, in string drv_name,
                           in string implementation_specific_data)
                                     raises(QueryManagerNotFound,
                                            CosQuery::QueryProcessingError);

     /**
      * shutdown query service.
      **/
     void shutdown();

    };

};

#endif
\end{verbatim}
\subsection{RC.idl}
\begin{verbatim}
#ifndef __RC_IDL
#define __RC_IDL
/*
 * definitions and pseudo-interfaces for custom Record Marshalling.
 * (C) Ruslan Shevchenko <Ruslan@Shevchenko.Kiev.UA>, 1999
 * (C) GradSoft, 2001
 * $Id: ProgrammingGuide_eng.tex,v 1.26 2002-12-05 12:01:22 kav Exp $
 */

#ifndef __COSQUERYCOLLECTION_IDL
#include <CosQueryCollection.idl>
#endif


#pragma prefix "gradsoft.kiev.ua"

/**
 * pseudo-interfaces for custom Record Masrshalling
 * The main entity is: RC-coded octet sequence.
 * We provide 2 pseudo-interfaces: RCReader and RCWriter
 * for reading/writing from/to RCSeq. 
 **/
module RC
{

///
typedef sequence<octet> OctetSeq;
//typedef CosQueryCollection::Record Record;
///
typedef sequence<CosQueryCollection::Record> RecordSeq;

//typedef CosQueryCollection::Decimal Decimal;

/**
 * throwed, when Reader discovered error in OctSeq. 
 **/
exception BadOctSeq
{
 /**
  * position of read failure (in bytes).
  **/
 long   pos;    
 /**
  * what was happened ?
  **/
 string reason; 
};

///
exception TypeNotImplemented
{
 ///
 CosQueryCollection::FieldType fieldType;
};

///
exception FieldValueIsNull {};
///
exception InvalidPosition {};

/**
 * header of RC-coded octet sequence.
 **/
struct RCHeader
{
  ///
  octet version;
  /// number of records in sequence.
  /// (if -1, than number of records is unknown).
  long  nRecords;
  /// number of fileds in one record.
  unsigned long nFields;
};

///
exception InvalidHeadData {};

/**
 * this pseudointerface must be mapped to RCWriter static class
 * in host language.
 */
interface Writer // pseudo
{
  
  /**
   *write header of Octet Sequence to octSeq_.
   *@param nRecords - number of records to be coded.
   *@param nFields - number of fields in one record.
   *@param pos - position (input really ignored, on output it
   * is settet to first position after header).
   *@param octSeq_ - sequence, in which we code.
   **/
  void  writeHeader(in long nRecords, in unsigned long nFields,
                    inout unsigned long pos, inout OctetSeq octSeq_)
                               raises (InvalidHeadData);

  ///
  void  writeHead(inout unsigned long pos, inout OctetSeq octSeq_)
                               raises (InvalidHeadData);


  ///
  void  writeRecord(in CosQueryCollection::Record record,
                     inout unsigned long pos,
                     inout OctetSeq octSeq_ )
                                           raises(TypeNotImplemented);

  ///
  void  writeRecordSeq(in RecordSeq recordSeq_)
                                           raises(TypeNotImplemented);

  ///
  void  writeBoolean(in boolean value, inout unsigned long pos,
                                       inout OctetSeq octSeq_);
                                          
  ///
  void  writeChar(in char value, inout unsigned long pos,
                                       inout OctetSeq octSeq_);

  ///
  void  writeShort(in short value, inout unsigned long pos,
                                       inout OctetSeq octSeq_);

  ///
  void   writeLong(in long value, inout unsigned long pos, inout OctetSeq octSeq_);

  ///
  void   writeULong(in unsigned long value, inout unsigned long pos, 
                                            inout OctetSeq octSeq_);

  ///
  void   writeFloat(in float value, inout unsigned long pos,
                                            inout OctetSeq octSeq_);

  ///
  void   writeDouble(in float value, inout unsigned long pos,
                                            inout OctetSeq octSeq_);
  
  ///
  void   writeString(in string value, inout unsigned long pos,
                                            inout OctetSeq octSeq_);

  ///
  void   writeWString(in wstring value, inout unsigned long pos,
                                            inout OctetSeq octSeq_);

  ///
  void   writeObject(in Object value, inout unsigned long pos,
                                            inout OctetSeq octSeq_);

  ///
  void   writeDecimal(in CosQueryCollection::Decimal value, 
                                    inout unsigned long pos,
                                            inout OctetSeq octSeq_);

  ///
  void   writeRaw(in OctetSeq value, inout unsigned long pos,
                                            inout OctetSeq octSeq_);

  ///
  void   writeDateTime(in CosQueryCollection::DateTime value, 
                       inout unsigned long pos,
                       inout OctetSeq octSeq_);

  ///
  void   writeFieldValue(in CosQueryCollection::FieldValue value,
                         inout unsigned long pos, inout OctetSeq octSeq_); 

  ///
  void   writeNullField(inout unsigned long pos, inout OctetSeq octSeq_);

  ///
  void   writeBooleanField(in boolean value,
                           inout unsigned long pos, inout OctetSeq octSeq_);

  ///
  void   writeCharField(in char value,
                           inout unsigned long pos, inout OctetSeq octSeq_);

  ///
  void   writeOctetField(in char value,
                           inout unsigned long pos, inout OctetSeq octSeq_);

  ///
  void   writeShortField(in short value,
                           inout unsigned long pos, inout OctetSeq octSeq_);

  ///
  void   writeUShortField(in unsigned short value,
                           inout unsigned long pos, inout OctetSeq octSeq_);

  ///
  void   writeLongField(in long value,
                           inout unsigned long pos, inout OctetSeq octSeq_);

  ///
  void   writeULongField(in unsigned long value,
                           inout unsigned long pos, inout OctetSeq octSeq_);

  ///
  void   writeFloatField(in float value,
                           inout unsigned long pos, inout OctetSeq octSeq_);

  ///
  void   writeDoubleField(in double value,
                           inout unsigned long pos, inout OctetSeq octSeq_);

  ///
  void   writeStringField(in string value,
                           inout unsigned long pos, inout OctetSeq octSeq_);

  ///
  void   writeObjectField(in Object value,
                           inout unsigned long pos, inout OctetSeq octSeq_);

  ///
  void   writeDecimalField(in CosQueryCollection::Decimal value,
                           inout unsigned long pos, inout OctetSeq octSeq_);

  ///
  void   writeRawField(in OctetSeq value,
                           inout unsigned long pos, inout OctetSeq octSeq_);

  ///
  void   writeLongRawField(in OctetSeq value,
                           inout unsigned long pos, inout OctetSeq octSeq_);
  ///
  void   writeBlob(in CosQueryCollection::Blob value,
                           inout unsigned long pos, inout OctetSeq octSeq_);
  ///
  void   writeClob(in CosQueryCollection::Clob value,
                           inout unsigned long pos, inout OctetSeq octSeq_);
  ///
  void   writeWclob(in CosQueryCollection::Wclob value,
                           inout unsigned long pos, inout OctetSeq octSeq_);

  ///
  OctetSeq  copyStream(in unsigned long from_pos, in unsigned long to_pos,
                     in OctetSeq octSeq_)
                           raises(InvalidPosition);

};

/**
* this pseudointerface must be mapped to RCReader static class
* in host language.
*/
interface Reader
{

  ///
  void    readHeader(inout RCHeader header, inout unsigned long pos,
                                            in OctetSeq octSeq_)
                                           raises(BadOctSeq);

  ///
  CosQueryCollection::Record  readRecord(inout unsigned long pos, in OctetSeq octSeq_ )
                                           raises(BadOctSeq);

  ///
  RecordSeq  readRecordSeq(inout unsigned long pos, in OctetSeq octSeq_)
                                           raises(BadOctSeq);
   
  ///
  CosQueryCollection::FieldValue  readField(inout unsigned long pos,
                                                 in OctetSeq octSeq_)
                                           raises(BadOctSeq);

  /**
   * return true and skip null value, if return was null, otherwise
   * return false and not touch pos.
   */
  boolean   nextFieldIsNull(inout unsigned long pos, in OctetSeq octSeq_)
                                           raises(BadOctSeq);

  ///
  CosQueryCollection::FieldType nextFieldType(inout unsigned long pos,
                                               in OctetSeq octSeq_)
                                           raises(BadOctSeq);

  ///
  boolean   readBooleanField(inout unsigned long pos, in OctetSeq octSeq_)
                                       raises(BadOctSeq,FieldValueIsNull);

  ///
  void      readBooleanField_inout(inout boolean value,
                                 inout unsigned long pos, in OctetSeq octSeq_)
                                          raises(BadOctSeq,FieldValueIsNull);

  ///
  char      readCharField(inout unsigned long pos, in OctetSeq octSeq_)
                                          raises(BadOctSeq,FieldValueIsNull);

  ///
  void      readCharField_inout(inout char value,
                                  inout unsigned long pos, in OctetSeq octSeq_)
                                          raises(BadOctSeq,FieldValueIsNull);

  ///
  octet     readOctetField(inout unsigned long pos, in OctetSeq octSeq_)
                                         raises(BadOctSeq,FieldValueIsNull);

  ///
  short     readShortField(inout unsigned long pos, in OctetSeq octSeq_) 
                                         raises(BadOctSeq,FieldValueIsNull);

  ///
  unsigned short  readUShortField(inout unsigned long pos, in OctetSeq octSeq_) 
                                        raises(BadOctSeq,FieldValueIsNull);

  ///
  long      readLongField(inout unsigned long pos, in OctetSeq octSeq_)
                                         raises(BadOctSeq,FieldValueIsNull);

  ///
  unsigned long  readULongField(inout unsigned long pos, in OctetSeq octSeq_)
                                          raises(BadOctSeq,FieldValueIsNull);

  ///
  float     readFloatField(inout unsigned long pos, in OctetSeq octSeq_)
                                          raises(BadOctSeq,FieldValueIsNull);

  ///
  double    readDoubleField(inout unsigned long pos, in OctetSeq octSeq_)
                                          raises(BadOctSeq,FieldValueIsNull);

  ///
  string    readStringField(inout unsigned long pos, in OctetSeq octSeq_)
                                          raises(BadOctSeq,FieldValueIsNull);

  ///
  Object  readObjectField(inout unsigned long pos, in OctetSeq octSeq_)
                                          raises(BadOctSeq,FieldValueIsNull);

  ///
  CosQueryCollection::Decimal readDecimalField(
                                  inout unsigned long pos,
                                  in OctetSeq octSeq_
                              )raises(BadOctSeq,FieldValueIsNull);
  
  ///
  CosQueryCollection::Decimal readNumericField(
                                  inout unsigned long pos,
                                  in OctetSeq octSeq_
                              )raises(BadOctSeq,FieldValueIsNull);

  ///
  CosQueryCollection::DateTime readDateTimeField(
                                   inout unsigned long pos,
                                   in OctetSeq octSeq_
                               )raises(BadOctSeq,FieldValueIsNull);

  ///
  OctetSeq  readRawField(inout unsigned long pos, in OctetSeq octSeq_)
                                          raises(BadOctSeq,FieldValueIsNull);

  ///
  void    readRawField_inout(inout OctetSeq value, 
                             inout unsigned long pos, in OctetSeq octSeq_)
                                          raises(BadOctSeq,FieldValueIsNull);

  ///
  OctetSeq  readLongRawField(inout unsigned long pos, in OctetSeq octSeq_)
                                          raises(BadOctSeq,FieldValueIsNull);

  ///
  void    readLongRawField_inout(inout OctetSeq value, 
                             inout unsigned long pos, in OctetSeq octSeq_)
                                          raises(BadOctSeq,FieldValueIsNull);

  ///
  string  readLongStringField(inout unsigned long pos, in OctetSeq octSeq_)
                                          raises(BadOctSeq,FieldValueIsNull);

  ///
  void  readLongStringField_inout(inout string value,
                                  inout unsigned long pos, in OctetSeq octSeq_)
                                          raises(BadOctSeq,FieldValueIsNull);
  ///
  void   readBlob(inout CosQueryCollection::Blob value,
                           inout unsigned long pos, in OctetSeq octSeq_)
                                          raises(BadOctSeq,FieldValueIsNull);
  ///
  void   readClob(inout CosQueryCollection::Clob value,
                           inout unsigned long pos, in OctetSeq octSeq_)
                                          raises(BadOctSeq,FieldValueIsNull);
  ///
  void   readWclob(inout CosQueryCollection::Wclob value,
                           inout unsigned long pos, in OctetSeq octSeq_)
                                          raises(BadOctSeq,FieldValueIsNull);


};


};

#endif
\end{verbatim}



\section{RC-coding specifications}

\begin{verbatim}
RCStream:: Version, RecordArray

Version:: 0x01

RecordArray :: NumberOfRecords[4],  RecordHeader , RecordData<1..infinity> ;

Record :: RecordHeader,RecordData;

RecordHeader :: NumberOfFields[1];

RecordData :: FieldBlock<NumberOfFields> ;

FieldBlock ::  DataType , DataValue;

DataType ::
 	       TypeNull	     0x00
	|      TypeBoolean   0x01
	|      TypeChar      0x02
	|      TypeOctet     0x03
	|      TypeShort     0x04
	|      TypeUShort    0x05
	|      TypeLong      0x06
	|      TypeULong     0x07
	|      TypeFloat     0x08
	|      TypeDouble     0x09
	|      TypeString     0x0A
	|      TypeObject     0x0B
	|      TypeAny        0x0C
	|      TypeSmallInt   0x0D
	|      TypeInteger    0x0E
	|      TypeDecimal    0x0F
	|      TypeNumeric    0x10
	|      TypeRaw        0x11
	|      TypeLongRaw    0x12
        |      TypeLongString 0x13
        |      TypeWStrint    0x14
        |      TypeDateTime   0x15
    ;	

DataValue  :: 
               ValueNull[0],
	|      ValueBoolean[1] 
	|      ValueChar[1]
	|      ValueWchar[2]
	|      ValueShort[2] // network order
	|      ValueUShort[2] // network order
	|      ValueLong[4] // network order
	|      ValueULong[4] // network order
	|      ValueFloat[4]     // network order
	|      ValueDouble[8]    // network order
	|      ValueString  
	|      ValueOctets    
	|      ValueWString
	|      ValueOctet[1]
        |      ValueDecimal
        |      ValueAny 
        |      ValueObject
        |      ValueDateTime
    ;

ValueDecimal:: ValueLong, ValueLong, ValueRaw
          // precision, scale, value

ValueDateTime:: ValueShort, ValueOctet, ValueOctet, ValueOctet, ValueOctet, ValueOctet
        //        year    ,   month   ,   day     ,   hour    ,  minute   , second

ValueString :: Length[4] //network order ,ValueChar<Length> ;

ValueWString :: Length[4] //network order ,ValueWChar<Length> ;

ValueOctets :: Length[4] //network order ,ValueOctet<Length> ;
 
ValueAny :: Length[4], TypeCode id as String, value as OctetSeq
                
ValueObject :: Length[4], GIOP ObjectReference

\end{verbatim}

\section{Grammar of collection SQL queries }

\label{Collection-select-grammar}

\begin{verbatim}
%%

startTerm: query
          ;

query:
   SELECT select_opt selection table_expr opt_order_by_clause opt_key_clause
        |
         was_error
       ;

select_opt:
          /* empty */
	|
	 ALL
	|
         DISTINCT
        ;

selection:
	select_scalar_expr_comma_list
	|
	'*'
        ;


table_expr:
        from_clause
          opt_where_clause 
            opt_group_by_clause 
             opt_having_clause  
           ;

from_clause:
         FROM table_ref_commalist
         |
         FROM '(' query ')' 
         ;

opt_where_clause:
              /* empty */
             |
              WHERE search_condition
             ;

opt_group_by_clause:
                  /* empty */
                 |
                  GROUP BY column_ref_commalist
                 ;
 
opt_having_clause:
                   /* empty */
                  |
                   HAVING search_condition
                  ;

opt_key_clause:
                /* empty */
                |
                 WITH KEY selection
                ;


opt_order_by_clause:
                  /* empty */
                  |
                   ORDER BY ordering_spec_commalist
                  ;


search_condition:
                  search_condition OR search_condition
                 |
                  search_condition AND search_condition
                 |
                  NOT search_condition
                 |
                  '(' search_condition ')'
                 |
                  predicate
                 |
                  '@'
                 ;

predicate:
           comparison_predicate
          |
           between_predicate
          |
           like_predicate
          |
           test_for_null
          |
           in_predicate
          |
           all_or_any_predicate
          |
           existence_predicate
          ;

comparison_predicate:
	           scalar_expr LESS  subquery
                  |
                   scalar_expr LESS  scalar_expr
                  |
	           scalar_expr LESS_EQ  subquery
                  |
                   scalar_expr LESS_EQ  scalar_expr
                  |
                   scalar_expr GT  scalar_expr
                  |
                   scalar_expr GT  subquery
                  |
                   scalar_expr GT_EQ  scalar_expr
                  |
                   scalar_expr GT_EQ  subquery
                  |
                   scalar_expr EQ subquery  
                  |
                   scalar_expr EQ scalar_expr 
                  |
                   scalar_expr NEQ subquery
                  | 
                   scalar_expr NEQ scalar_expr
                  ;
                   
test_for_null:
	scalar_expr IS NOT DBNULL
        |
        scalar_expr IS DBNULL
        ;
 

in_predicate:
        scalar_expr NOT IN  subquery 
       |
        scalar_expr IN  subquery 
       |
        scalar_expr NOT IN '(' atom_commalist ')'
       |
        scalar_expr IN '(' atom_commalist ')'
       ;

all_or_any_predicate:
            scalar_expr LESS any_all_some subquery
           |
            scalar_expr LESS_EQ any_all_some subquery
           |
            scalar_expr GT any_all_some subquery
           | 
            scalar_expr GT_EQ any_all_some subquery
           |
            scalar_expr EQ any_all_some subquery
           |
            scalar_expr NEQ any_all_some subquery
           ;

any_all_some:
           ANY 
          |
           ALL
          |
           SOME
          ;

existence_predicate:
            EXISTS subquery
           ;

between_predicate:
                   scalar_expr NOT BETWEEN scalar_expr AND scalar_expr
                  |
                   scalar_expr BETWEEN scalar_expr AND scalar_expr
                  ;

like_predicate:
	        scalar_expr NOT LIKE atom opt_escape
               |
                scalar_expr LIKE atom opt_escape
               ;        
 
opt_escape:
	  /* empty */
           { $$=NULL; }
         |
          ESCAPE atom
         ;

ordering_spec_commalist:
         ordering_spec
         |
         ordering_spec_commalist ',' ordering_spec
         ;

ordering_spec:
           INTNUM opt_ordering
          |
           column_ref opt_ordering
          ;

opt_ordering:
         /* empty */
         |
          ASC
         |
          DESC
         ; 

subquery:
         '(' SELECT select_opt selection table_expr ')'
         ;

select_scalar_expr_comma_list:
                        select_scalar_expr
                        |
                         select_scalar_expr_comma_list ',' select_scalar_expr
                        ;

select_scalar_expr:
                   '@'
                   |
                    scalar_expr
                   ;



scalar_expr:
	scalar_expr '+' scalar_expr
       |
        scalar_expr '-' scalar_expr
       |
        scalar_expr '*' scalar_expr
       |
        scalar_expr '/' scalar_expr
       |
        scalar_expr CONCAT_OP scalar_expr 
       |
        '-' scalar_expr  %prec UMINUS
       |
        '+' scalar_expr  %prec UMINUS
       |
        atom
       |
        column_ref
       |
        function_ref
       |
        '(' scalar_expr ')'
       ;

table_ref_commalist:
             table_ref
            |

             table_ref_commalist ',' table_ref
            ;

atom_commalist:
        atom
       |
        atom_commalist ',' atom
       ;

atom:
	parameter_ref
       |
        literal
       |
        { $$=yy_createConstStringTerm("USER"); }
       ;

column_ref_commalist:
        column_ref
        |
        column_ref_commalist ',' column_ref
        ;

column_ref:
        IDENT
       |
        IDENT '.' IDENT
       |
        IDENT '.' IDENT '.' IDENT
       ; 

function_ref:
           compound_name '(' '*' ')'
          | 
           compound_name '(' DISTINCT column_ref ')'
          |
           compound_name '(' ALL scalar_expr ')'
          |
           compound_name '(' scalar_expr ')'
          ; 

compound_name:
          IDENT
         |
          IDENT '.' IDENT
         ;

table_ref:
          compound_name 
         |
          compound_name  IDENT
         |
          '@'
         ;

parameter_ref:
	parameter
        |
        parameter parameter
        |
        parameter INDICATOR parameter
        ;


parameter:
          ':' IDENT
         ;

literal:
        STRING
       |
        INTNUM
       |
        APPROXNUM
       ; 

was_error: 
          error
         ;
\end{verbatim}

\subsection{List Of Changes}

\begin{itemize}
  \item 24.01.2002 routine of operating with LOB described
  \item 21.01.2002 examples code review, InterBase to CORBA type relations described
  \item 19.06.2001 - 1.0.2 initialization is described.
  \item 06.06.2001 - review.
  \item 08.05.2001 review, adding of applications.
  \item 16.04.2001 first preview.
\end{itemize}

\bibliography{Bib}


\end{document}

